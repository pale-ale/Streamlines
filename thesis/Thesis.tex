% !TeX root = Thesis.tex
\documentclass[11pt]{mimosis}
\synctex=1

% \PassOptionsToClass{14pt}{scrbook}
\usepackage{metalogo}

\usepackage{textcomp}
\usepackage{gensymb}

\usepackage[pass]{geometry}

%%%%%%%%%%%%%%%%%%%%%%%%%%%%%%%%%%%%%%%%%%%%%%%%%%%%%%%%%%%%%%%%%%%%%%%%
% Some of my favorite personal adjustments
%%%%%%%%%%%%%%%%%%%%%%%%%%%%%%%%%%%%%%%%%%%%%%%%%%%%%%%%%%%%%%%%%%%%%%%%
%
% These are the adjustments that I consider necessary for typesetting
% a nice thesis. However, they are *not* included in the template, as
% I do not want to force you to use them.

% This ensures that I am able to typeset bold font in table while still aligning the numbers
% correctly.
\usepackage{etoolbox}

\usepackage[binary-units=true]{siunitx}
\DeclareSIUnit\px{px}

\sisetup{%
  detect-all           = true,
  detect-family        = true,
  detect-mode          = true,
  detect-shape         = true,
  detect-weight        = true,
  detect-inline-weight = math,
}

%%%%%%%%%%%%%%%%%%%%%%%%%%%%%%%%%%%%%%%%%%%%%%%%%%%%%%%%%%%%%%%%%%%%%%%%
% Hyperlinks & bookmarks
%%%%%%%%%%%%%%%%%%%%%%%%%%%%%%%%%%%%%%%%%%%%%%%%%%%%%%%%%%%%%%%%%%%%%%%%

\usepackage[%
  colorlinks = true,
  citecolor  = BrickRed,
  linkcolor  = BrickRed,
  urlcolor   = BrickRed,
  % pdftex,
  pdfauthor={Guybrush Threepwood},
  pdftitle={Reflections on the correlation between the tendency to procrastinate and the likelihood to be awarded a doctoral degree},
  % pdfsubject={The Subject},
  % pdfkeywords={Some Keywords},
  % pdfproducer={Latex with hyperref, or other system},
  % pdfcreator={pdflatex, or other tool}
  final,
  ]{hyperref}

\usepackage{bookmark}


%%%%%%%%%%%%%%%%%%%%%%%%%%%%%%%%%%%%%%%%%%%%%%%%%%%%%%%%%%%%%%%%%%%%%%%%
% Bibliography
%%%%%%%%%%%%%%%%%%%%%%%%%%%%%%%%%%%%%%%%%%%%%%%%%%%%%%%%%%%%%%%%%%%%%%%%
%
% I like the bibliography to be extremely plain, showing only a numeric
% identifier and citing everything in simple brackets. The first names,
% if present, will be initialized. DOIs and URLs will be preserved.

\usepackage[%
  autocite      = plain,
  backend       = biber,
  doi           = false,
  url           = true,
  giveninits    = true,
  hyperref      = true,
  maxbibnames   = 99,
  maxcitenames  = 99,
  sortcites     = true,
  style         = alphabetic,
  citestyle     = alphabetic,
  maxalphanames = 4,           % max number of authors before it becomes [First+12]
  backref       = true,
  ]{biblatex}


\input{bibliography-mimosis}
\bibliography{Thesis}

%%%%%%%%%%%%%%%%%%%%%%%%%%%%%%%%%%%%%%%%%%%%%%%%%%%%%%%%%%%%%%%%%%%%%%%%
% Fonts
%%%%%%%%%%%%%%%%%%%%%%%%%%%%%%%%%%%%%%%%%%%%%%%%%%%%%%%%%%%%%%%%%%%%%%%%

\ifxetexorluatex
  %\setmainfont{Minion Pro}
  \usepackage{microtype}
\else
  %\usepackage[osf,lining]{ebgaramond}  
  \usepackage[scale=0.7]{sourcecodepro}  
\fi


\usepackage{amsmath}
\usepackage{mathpazo}
\usepackage{lettrine}
\usepackage{tikz}
\usepackage{wrapfig}
\usepackage{pgfplots}
\usepackage[justification=centering]{caption}
\pgfplotsset{compat=newest}

% \usepackage{minipage}

\usepackage{makeidx}
\makeindex



% \newacronym[description={Principal component analysis}]{PCA}{PCA}{principal component analysis}
% \newacronym                                            {SNF}{SNF}{Smith normal form}
% \newacronym[description={Topological data analysis}]   {TDA}{TDA}{topological data analysis}

% \makeindex
% \makeglossaries

%%%%%%%%%%%%%%%%%%%%%%%%%%%%%%%%%%%%%%%%%%%%%%%%%%%%%%%%%%%%%%%%%%%%%%%%
% Incipit
%%%%%%%%%%%%%%%%%%%%%%%%%%%%%%%%%%%%%%%%%%%%%%%%%%%%%%%%%%%%%%%%%%%%%%%%
 
\usepackage{mathtools}
\usepackage{amssymb}
\usepackage{siunitx}

\usepackage{blindtext}

% Corrects \autoref{}: chapter -> Chapter, section -> Section, subsection -> Section
\addto\extrasenglish{%
  \renewcommand{\chapterautorefname}{Chapter}%
  \renewcommand{\sectionautorefname}{Section}%
  \renewcommand{\subsectionautorefname}{Section}%
}

\begin{document}

\frontmatter
  %!TEX root = ../Thesis.tex

\begin{titlepage}
  \centering % Center all text
  \vspace*{\baselineskip} % White space at the top of the page

  %\rule{\textwidth}{1.6pt}\vspace*{-\baselineskip}\vspace*{2pt} % Thick horizontal line
  %\rule{\textwidth}{0.4pt}\\[1.0\baselineskip] % Thin horizontal line

  {\huge Time-Coherent Streamline Placement}\\[0.2\baselineskip] % Title

  %\rule{\textwidth}{0.4pt}\vspace*{-\baselineskip}\vspace{3.2pt} % Thin horizontal line
  %\rule{\textwidth}{1.6pt}\\ % Thick horizontal line

  \vspace*{\baselineskip}

  {\Large Bachelor's Thesis\\[\baselineskip]} % Tagline(s) or further description
  \vspace*{\baselineskip}

  {\LARGE Alexander Baucke\\[\baselineskip]} % Editor list  

  \vspace*{\baselineskip} % Whitespace between location/year and editors

  Supervisor\\
  {\large  Prof.\,Dr.\,Filip Sadlo\\[\baselineskip]} % Editor list

  \vfil

  Heidelberg, Germany,  \today \par % Location and year

  \vspace*{\baselineskip}

  {\itshape Faculty of Mathematics and Computer Science\par} % Editor affiliation
  {\itshape Heidelberg University\par} % Editor affiliation
\end{titlepage}

\cleardoublepage

  % %!TEX root = ../Thesis.tex

\selectlanguage{english}

\begin{titlepage}
  \begin{center}
    \textsc{\huge Dissertation}
                \vskip 1cm
                \begin{large}
                  submitted to the\\[0.50cm]
                  \begin{Large}
                    \textsc{Combined Faculty for the\\Natural Sciences and Mathematics}\\[0.50cm]
                  \end{Large}
                  of\\[0.50cm]
                  \begin{Large}
                    \textsc{Heidelberg University, Germany}\\[0.50cm]
                  \end{Large}
                  for the degree of \\[0.5cm] 
                  Doctor of Natural Sciences
                \end{large}
    %
    \vfill
    %
    \begin{large}
                  put forward by\\[0.5cm]
                  \begin{LARGE}
                    \textbf{Guybrush Threepwood}, M.Sc.
                  \end{LARGE}\\[0.5cm]
                  born in Ankh-Morpork, Discworld
    \end{large}
    %
    \vskip 2cm
    %
    \begin{small}
      Heidelberg\\
      Marchtember 2033
    \end{small}
  \end{center}
\end{titlepage}

\selectlanguage{english}

\begin{titlepage}
  %
  \phantom{}
  \vfill 
  %
  \begin{center}
    \begin{singlespace*}
      \begin{Huge}
          Reflections on the correlation between the tendency to procrastinate and the likelihood to be awarded a doctoral degree\par
      \end{Huge}
      %
      \vskip 0.25cm
      \emph{by}
      \vskip 0.25cm
      %
      \textsc{Guybrush Threepwood}\par
    \end{singlespace*}
  \end{center}
  %
  \vfill
  %
  \begin{singlespace*}
    Advisor:            Prof.\,Dr.\,Filip Sadlo \\[.5cm]
    Oral examination: \rule{4cm}{0.15mm}
  \end{singlespace*}
\end{titlepage}

\newpage
\null
\thispagestyle{empty}
\newpage
  % for PhD thesis
  \pagestyle{empty}
  %!TEX root = ../Thesis.tex

\section*{Declaration of Authorship}

% Official declaration of authorship from the Institute for Computer Science in german from https://www.informatik.uni-heidelberg.de/studium/thesis-erklaerung?lang=de
%Hiermit versichere ich, dass ich die Arbeit selbst verfasst und keine anderen als die angegebenen Quellen und Hilfsmittel benutzt und wörtlich oder inhaltlich aus fremden Werken Übernommenes als fremd kenntlich gemacht habe. Ferner versichere ich, dass die übermittelte elektronische Version in Inhalt und Wortlaut mit der gedruckten Version meiner Arbeit vollständig übereinstimmt. Ich bin einverstanden, dass diese elektronische Fassung universitätsintern anhand einer Plagiatssoftware auf Plagiate überprüft wird.

% Official declaration of authorship from the Institute for Computer Science from https://www.informatik.uni-heidelberg.de/studium/thesis-erklaerung?lang=en
I hereby certify that I have written the work myself and that I have not used any sources or aids other than those specified and that I have marked what has been taken over from other people's works, either verbatim or in terms of content, as foreign. I also certify that the electronic version of my thesis transmitted completely corresponds in content and wording to the printed version. I agree that this electronic version is being checked for plagiarism at the university using plagiarism software.

% Old version
%I hereby declare that the thesis submitted is my own unaided work. All direct or indirect sources used are acknowledged as references. The principles and recommendations \enquote{Verantwortung in der Wissenschaft} of Heidelberg University have been followed.
\vspace{5cm}
\noindent\rule[0.5ex]{8em}{0.5pt} \hfill \rule[0.5ex]{10em}{0.5pt}\\
\noindent first and last name \hfill city, date and signature  % remove for PhD thesis
  %!TEX root = ../Thesis.tex

\begin{center}
  \textsc{Abstract}
\end{center}
%
\noindent
Vector field visualizations are used in many fields like aerospace engineering, fluid dynamics, or physics.
A common graphical representation of such fields are instantaneous integral lines, called streamlines.
The placement of such streamlines for a continuous and steady vector field is subject of contemporary research,
with a multitude of algorithms that can generate such streamline placements in an optimal way w.r.t different criteria.
This thesis's focus will be on the time-coherent placement of streamlines, i.e. minimizing streamline movement during changes of the underlying continuous, but unsteady vector field.
We present an iterative procedure that starts by seeding streamlines with a greedy algorithm for a single timestep.
The seeds are then optimized to make them time coherent for as many lines a possible.
The performance and complexity using different streamline seeding and modification strategies are examined and compared.
Finally some limitiations, possible imporovements, and ideas for future work will be listed. 
\cleardoublepage


  %!TEX root = ../Thesis.tex

\begin{center}
  \textsc{Zusammenfassung}
\end{center}
%
\selectlanguage{ngerman}
Ziel dieser Arbeit ist das Erreichen von Zeitkohärenz bei Stromlinien
in zeitabhängigen Vektorfeldern, um deren Darstellung durch z.B. Animationen zu verbessern.
Wir beginnen mit einer Definition für Zeitkohärenz, die im Wesentlichen auf die
Bewegung einer Stromlinie von einem Zeitschritt zum nächsten gestützt ist.
Wenn diese Bewegung groß ausfällt bzw. sich die Stromlinie viel bewegt hat,
bezeichnen wir dies als ``schlechte Zeitkohärenz''.
Wir zeigen, warum zeitkohärentes Verhalten wichtig ist für z.B. Animationen,
und entwickeln eine prototypische Implementierung, deren Parametrisierung
wir dann anhand einiger Beispiele bestimmen.
Unser Algorithmus basiert auf einer durch Bild-geleiteten Implementierung von Greg Turk und David Banks,
den wir durch einige Änderungen an den zentralen Komponenten zeitkohärent gemacht haben.
Im Anschluss werden einige Datensätze damit ausgewertet und mit anderen einfachen Algorithmen verglichen;
außerdem beschreiben wir einige Limitierungen sowie Ideen für Verbesserungen.
Zum Schluss wird eine kurze Komplexitätsanalyse durchgeführt und die Performance untersucht.
\selectlanguage{english}
\cleardoublepage

  \tableofcontents

\mainmatter
  \pagestyle{scrheadings}

  %!TEX root = ../Thesis.tex

%%%%%%%%%%%%%%%%%%%%%%%%%%%%%%%%%%%%%%%%%%%%%%%%%%%%%%%%%%%%%%%%%%%%%%%%
\chapter{Introduction}
%%%%%%%%%%%%%%%%%%%%%%%%%%%%%%%%%%%%%%%%%%%%%%%%%%%%%%%%%%%%%%%%%%%%%%%%

For steady fields, there are several papers showing different strategies to generate streamlines according to different criteria.
Commonly preferred attributes are high streamline length, and uniform line density, which are often labelled as "coherence".
Both of which greatly enhance the information uptake by preventing visual clutter;
thereby allowing a focus on more important features and characteristics of the field at hand.
As soon as a change in the field is introduced, however, these methods are no longer optimal when examined over the entire time span.
Therefore, we will start by introducing a new criterion termed "temporal coherence" (opposed to the aforementioned spatial coherence).
This essentially defines how lines move through time; low coherence means a lot of movement. For a more detailed explanation, see the next section.
The procedure outlined by this paper has a simple modus operandi:
We start by generating a spacially coherent streamline structure for every time step using a greedy algorithm.
Then we start walking along the generated lines, trying to find those of high temporal cohrence, and keep them.
The others we try to optimize by moving their seeds around, or, if no sensible movement is possible, simply delete them.
In the final step, we fill the blank spaces left by the deletion according to the chosen seeding algorithm, and are finished.

\section{Problem Statement}
Creating animations containing streamlines is difficult because streamlines are not time-coherent when generated using conventional methods described in ...

\section{Proposed Solution}
By adding the time coherence constraint, animated streamline visualizations look much better and do not introduce aritfacts etc. etc.





  %%%%%%%%%%%%%%%%%%%%%%%%%%%%%%%%%%%%%%%%%%%%%%%%%%%%%%%%%%%%%%%%%%%%%%%%
\chapter{Related Work}
\label[section]{sec:relatedWork}
%%%%%%%%%%%%%%%%%%%%%%%%%%%%%%%%%%%%%%%%%%%%%%%%%%%%%%%%%%%%%%%%%%%%%%%%

Most of the work published in the field of streamline placement algorithms can be divided into three categories,
each having different focuses, strengths, and drawbacks.
Each category will be briefly addressed in the following sections.

\section{Image-Guided Algorithms}
The goal of this approach is to achieve a very uniform spacing of streamlines,
giving an almost "hand-drawn" appearance.
Generated images will usually have high visual quality,
at the risk of potentially missing or misrepresenting features and higher computational cost.

One of the first and most prominent examples in this category was created
by Greg Turk and David Banks~\cite{TurkBanks} in 1996,
which we have adopted as the base for our algorithm as well.
In their research paper, a function (called the ``Energy Function'')
is defined such that it maps an input image containing the potential streamlines to a scalar.
The scalar represents the quality of the image, roughly defined as the uniformity of the streamline spacing. 
Adding/moving/removing/resizing streamlines is done randomly,
at every step the energy function is used to determine whether a proposed change gets accepted or discarded.
The algorithm is finished when the energy function reaches a lower energy bound or the user stops it.
This idea was extended by \cite{https://doi.org/10.1111/j.1467-8659.2009.01352.x} to allow
streamline generation on 3D surfaces.
Another image-guided approach, which is also view-dependent for 3D fields, was created by
\cite{8093671}.


\section{Feature-Guided Algorithms}
Feature-guided algorithms examine the underlying field structure when placing seeds.
They search for critical points or patterns in the field and then seed around them,
capturing them in much higher detail.
The resulting images inherently represent the critical points much better,
sometimes at the cost of visual appeal compared to image-guided algorithms.
\cite{Verma} is one of the first algorithms in this category, extracting field topology
and then applying a fixed pattern of seeds for the different structures to capture them
as accurately as possible.
\cite{1532832} developed a farthest-point seeding strategy using Delaunay triangulation
for fast seed positioning.
\cite{wu2009topology} picks up on this idea, improving it to generate better images still adhering
to the underlying topology.

\section{Greedy Algorithms}
Greedy algorithms do not use a global guiding principle for streamline placement.
We used an approach similar to \cite{Jobard} as a prototype,
placing seed candidates along a generated line and attempting to grow streamlines 
from them until the space is filled.
\cite{4015453} later improved this algorithm's speed by an order of magnitude.
\cite{Mattausch} extended this principle into 3D,
adding several features to make it more suitable for user interaction and 3D rendering.






  %%%%%%%%%%%%%%%%%%%%%%%%%%%%%%%%%%%%%%%%%%%%%%%%%%%%%%%%%%%%%%%%%%%%%%%%
\chapter{Fundamentals}
\begin{figure}[ht]
    \centering
    \begin{subfigure}{.5\textwidth}
        \centering
        \includegraphics[scale=.1]{figures/WavesArrowField.png}
        \caption{A vector field defined by $u(x,y) = (1, \sin(x))^T$, visualized using arrows placed on a regular grid.}
    \end{subfigure}
    \label[figure]{fig:fundamentals_1}
\end{figure}
\noindent In this section, we describe the elementary components used in the remainder of this thesis.
Since this work is about the placement of streamlines in vector fields,
we start with the fundamentals of the field \textit{vector field visualization} in \Cref{sec:VFV}.
Our algorithm uses an image-guided base, hence we will also include some topics from the area of \textit{image processing} in \Cref{sec:IP}.
We conclude this chapter with a brief overview of the roots of unity in \Cref{sec:ROU},
because they are used in the development of an initial prototype in \Cref{sec:3D}.
%%%%%%%%%%%%%%%%%%%%%%%%%%%%%%%%%%%%%%%%%%%%%%%%%%%%%%%%%%%%%%%%%%%%%%%%
\section{Vector Field Visualization}
\label[section]{sec:VFV}
\paragraph*{Vector Field} A vector field represents how vectorized elements act over a spatial domain.
Concerning vector fields representing flow, this means that for every point in a domain, we can obtain the force acting at that point.

More formally, we can define a vector field as a map from $n$-dimensional scalars to $m$-dimensional scalars.
We can write it as an $n$-$m$-valued function, and in this thesis, we will only care about cases of $n=m$ in two and three dimensions.
There are several ways to obtain such fields, one is via an algebraic definition such as $u(x,y) = (1, \sin(x))^T$, giving us a field like the one seen in \Cref{fig:fundamentals_1}.
If we want our force to not only depend on spatial input but also on another scalar like a time component, we write this as $u(x,y,t)$ for the 2D case.       
We call vector fields \textit{steady} if they are not time-dependent; otherwise, we refer to them as \textit{unsteady}.
Another distinction is \textit{continuity}, which is analogous to the algebraic definition of other functions.
The fields in this work are all going to be continuous.

\paragraph*{Critical Points} A vector field can have points with distinct characteristics, called critical points.
In the 2D case, only four commonly used critical points exist, which we briefly describe here.
\begin{description}
    \item[Source]
    Given a field such as $u(x,y) = (x,y)$, at every point applies a force away from the origin.
    If we think about this as a non-compressible flow, this is equivalent to matter being created at the point $(0,0)$ and flowing outward.
    We refer to such a point as a \textit{source}.
    \item[Sink]
    Similarly, $u(x,y) = (-x,-y)$ would give us a \textit{sink} at $(0,0)$, equivalent to destroying liquid flowing inward.
    \item[Saddle]
    A saddle is an area where matter is pinched in one direction and stretched in another, e.g. in a field defined by $u(x,y) = (-x,y)$.
    \item[Periodic Orbit]
    $u(x,y) = (-y,x)$ creates circular paths around the origin where, after traveling a certain distance, a particle arrives at the point it started at.
    These critical points are called \textit{periodic orbits}.
\end{description}
\paragraph*{Streamlines}
Given a vector field $u$ and a point $P$, we can trace the movement of this point through $u$ by integrating over the field.
Intuitively, we can step through the field by choosing the next point $P_n = P_{n-1} + c \cdot u(P_{n-1})$, with $c$ being a step size scale.
If we do this an infinite number of times with $\pm c$ close to zero,
we end up with a set of points $S$ we have passed through, which defines the streamline.
$S$ has two notable properties:
\begin{itemize}
    \item Every point $P\in S$ inside this set has a velocity equal to $u(P)$.
    Hence, a streamline is tangent to the vector field at every point.
    \item No matter which point inside $S$ we use as $P_0$, we will always obtain the same set $S$ as its streamline.
    Therefore, any point inside $S$ is a potential \textit{seed} yielding the streamline $S$.
\end{itemize}
\newpage
\begin{figure}[t]
    \centering
    \begin{subfigure}{.29\textwidth}
        \centering
        \includegraphics[scale=.08]{figures/WavesStreamlines.png}
        \caption*{(a)}
    \end{subfigure}
    \begin{subfigure}{.29\textwidth}
        \centering
        \includegraphics[scale=.08]{figures/WavesStreamlinesBlur.png}
        \caption*{(b)}
    \end{subfigure}
    \begin{subfigure}{.3\textwidth}
        \centering
        \setlength\pgfplotswidth{1.5\textwidth}
        % \begin{tikzpicture}
%     \def\r(#1){abs(#1) / 2}%
%     \begin{axis}
%         \addplot[color=blue, ultra thick, samples=100, domain=-2.5:2.5]{
%             \r(x) < 1 ? 2*(\r(x))^3-3*(\r(x))^2+1 : 0
%         };
%         \addplot[color=red, dotted, ultra thick, domain=-2.5:2.5]{
%             2*(\r(x))^3-3*(\r(x))^2+1
%         };
%     \end{axis}
% \end{tikzpicture}
\begin{tikzpicture}[]
    \clip[use as bounding box] (0,0) rectangle (\pgfplotswidth,.55\pgfplotswidth);

    \def\r(#1,#2){(((#1)^2 + (#2)^2) / 2)^0.5}%
    \def\K(#1,#2){2 * \r(#1,#2)^3 - 3 * \r(#1,#2)^2 + 1}

    \begin{axis}[
        % xmin=-2.5,xmax=2,
        % ymin=-2.5,ymax=2,
        zmin=0, zmax=1,
        width=\pgfplotswidth,
        height=.8\pgfplotswidth,
        xticklabel=\empty,
        yticklabel=\empty,
        zticklabel=\empty,
        % axis line style={draw=none},
        % axis equal image,
    ]
    \addplot3[
        surf,
        domain=-2:2,
        samples=40
    ]
    % K(x,y) = 2r^3 - 3r^2 + 1 if r < 1 else 0
    % r = sqrt(x^2 + y^2) / R
    % R = desired radius, lets use 2
    % therefore we get:
    % r = sqrt(x^2 + y^2) / 2
    % K(x,y) = sqrt(x^2 + y^2) / 2 < 1 ? 2 * (sqrt(x^2 + y^2) / 2) ^ 3 - 3 * (sqrt(x^2 + y^2) / 2) ^ 2 + 1 : 0
    {\r(x,y) < 1 ? \K(x,y) : 0};% < 1 ? 2 * r(x,y) ^ 3 - 3 * r(x,y) ^ 2 + 1 : 0};
    \end{axis}
\end{tikzpicture}
% \pgfplotsset{width=7cm,compat=1.8}
% \pgfmathdeclarefunction{gauss}{2}{%
%     \pgfmathparse{1/(#2*sqrt(2*pi))*exp(-((x-#1)^2)/(2*#2^2))}%
% }
% \begin{tikzpicture}
%     \begin{axis}[every axis plot post/.append style={
%         ultra thick, samples=100, domain=-2.5:2.5, mark=none}]
%         \addplot {abs(x) < 2 ? 1.6 * gauss(0,2/3) : 0};
%     \end{axis}
% \end{tikzpicture}
% \begin{tikzpicture}
%     \begin{axis}[every axis plot post/.append style={
%         ultra thick, samples=100, domain=-12:12, mark=none}]
%         \addplot {abs(x) < 10 ? 8.4 * gauss(0,10/3) : 0};
%         \def\r(#1){abs(#1) / 10}%
%         \addplot[color=red, dotted, ultra thick, samples=100, domain=-12:12]{
%             \r(x) < 10 ? 2*(\r(x))^3-3*(\r(x))^2+1 : 0
%         };
%     \end{axis}
% \end{tikzpicture}
        \caption*{(c)}
    \end{subfigure}
    \caption{A set of streamlines (a) generated for the field in \Cref{fig:fundamentals_1}(a). (b) Low-pass version of the image after convolving it with the kernel shown in (c).}
    \label[figure]{fig:fundamentals_2}
\end{figure}
\paragraph*{Spatial Coherence}
    If we want to visualize a vector field, we want its features to be easily identifiable.
    At the same time, we do not want to introduce distractions or artifacts due to the visualization technique.
    The deciding factors of uniformity in streamline visualization are streamline length and density.
    Longer streamlines make for a smoother appearance, whereas many short lines tend to obfuscate and hinder the recognition of important features like the aforementioned critical points.
    Strong spatial coherence is shown in \Cref{fig:fundamentals_2}(a).

\section{Image processing}
\label[section]{sec:IP}
\paragraph{Convolution} A process often found in image- or signal processing.
A kernel (\Cref{fig:fundamentals_1}(c)) is applied to every pixel in an image,
affecting it and other surrounding pixels by adding or subtracting its value at that position. 

\paragraph{Blurring} A special type of convolution, making edges in an image less sharp.
Note the difference between black and white contrast for \Cref{fig:fundamentals_2}(a) and (b).

\section{Roots of Unity}
\label[section]{sec:ROU}
\begin{figure}[ht]
    \centering
    \begin{subfigure}[position]{.5\textwidth}
        \centering
        \setlength\pgfplotswidth{.9\textwidth}
        \begin{tikzpicture}
    \tikzset{
        pics/carc/.style args={#1:#2:#3}{
            code={
               \draw[pic actions] (#1:#3) arc(#1:#2:#3);
            }
        }
    }
    \begin{axis}[ 
        ticks=none,
        axis lines = middle,
        axis line style={->},
        ymin=-1.3, ymax=1.3,
        xmin=-1.3, xmax=1.3,
        xlabel={$1$},
        ylabel={$i$},
        width=\pgfplotswidth,
        height=\pgfplotswidth
        ]
        \draw (0,0) circle [radius=1, fill=white];
        \draw[very thick] (0,0) pic[red]{carc=0:72:2cm};
        \node[red] at (1,1) {$\frac{2\pi}{5}$};

        \draw[very thick, ->] (0,0) -- (1,0) node[midway, above right] {$n_0$};
        \draw[very thick, ->] (0,0) -- ( .309, .951) node[midway, right] {$n_1$};
        \draw[very thick, ->] (0,0) -- (-.809, .588) node[midway, above] {$n_2$};
        \draw[very thick, ->] (0,0) -- (-.809,-.588) node[midway, below] {$n_3$};
        \draw[very thick, ->] (0,0) -- ( .309,-.951) node[midway, right] {$n_4$};
    \end{axis}
\end{tikzpicture}

        \caption*{(a)}
    \end{subfigure}
    \caption*{The 5th roots of unity $n_0...n_4$ partition the unit circle equally}
\end{figure}
With $i$ being the complex number, we can use Euler's equation \[n_j = e^{ji2\pi/k}, j = 0, 1, ..., k-1\] to obtain $k$ numbers lying on the complex unit circle,
which are called the $k$-th roots of unity.
Notable properties are their length of exactly one, and that they divide the unit circle equally.
We can convert them to vectors in $\mathbb{R}^2$ using \[\vec{v_j} = (\operatorname{Re}(n_j), \operatorname{Im}(n_j))^T\]

% paar mehr beispiele, convolution etc, ausblick
% which algortimh?
  \chapter{Method}

At first, a heuristic criterion for temporal coherence between stream lines is defined.
Then ...


\begin{itemize}
    \item old implementation
    \item issues, why sth else was used
    \item turk and banks
    \item why turk and banks? (spacial coherence?)
    \item time coherence definition
\end{itemize}


\section{Temporal Coherence}
Temporal Coherence refers to how a vector field behaves through different time steps.
Intuitively, we consider areas within the field to be of high temporal coherence if the lines drawn on them are relatively stationary.
Vice versa, we can say that an area of high fluctuation will be of low temporal coherence.
A more formal definition employed in our algorithm is as follows:
Given a field $F$ and a starting point $S_0$ (called the "seed"), we can integrate over the field.
This yields a set of points $S^0$ which define a streamline containing every reached point, written as $S^0 = \int(S_0, F)$.
We can therefore assign a streamline to every point in our field (and vice versa).
Given $S_0$ and an unsteady field $F(t)$, compute for each time step $t_1...t_n$ the streamline $S^{0,t_i} = \int(S_0, F(t_i))$.
In order to convert these sets of lines to a scalar, we use the Hausdorff Distance $dist(S^i,S^j)$,
giving us the greatest minimal distance between any pair of two sets.
We can therefore create a map $coh(S_i, F(t)): max(dist(\int(S_i, F(t_k)), \int(S_i, F(t_l))))$,
sending each point in an unsteady vector field to a scalar, and thereby determining its temporal coherence.


\section{Technical details}
\subsection{Energy Measure}
The method used by Turk and Banks defines three important components to measure image quality as the sum of deviations of a low-pass image from a uniform greyscale target.
\begin{enumerate}
    \item The first component is a collection of (straight) line segments from each line,
    each of which can be converted to a line formula of the form $p = start + (end - start) * c$.
    The formula is then evaluated to obtain points as pixels where the low-pass filter in the 2nd listing is applied.
    In their paper, they call the implicit image obtained from the line segments' footprint $I$.
    \begin{equation*}
        I(x,y) = \begin{cases}
            1, \kern4em & \text{pixel lies on line}\\
            0,          & \text{else}
    \end{cases}
    \end{equation*}
    
    \item The second component is the low-pass filter $L$.
    It uses a kernel to generate the filtered image of a line.
    Given a falloff distance $R$ and $r=\sqrt{x^2+y^2} / R$, the kernel is defined as:
    \begin{equation*}
        K(x,y) = \begin{cases}
            2r^3 - 3r^2 + 1, \kern4em & r < 1\\
            0,               & r >= 1
        \end{cases}
    \end{equation*}
    For every pixel a line passes through, this kernel is applied additively, with its origin centered on the pixel containing the line.
    When applied consecutively along a line segment, the kernel will overlap and produce numbers from 1 to R for pixels close to the line.
    
    \item In order to determine the energy of the image generated by the kernel application, the following expression is used:
    \begin{equation*}
        E(I) = \int_x\int_y\left[(L\ast I)(x,y)-t\right]^2\,dx\,dy
    \end{equation*}
    With $t$ referring to the \textit{target brightness}, in their source code the number one is used.
\end{enumerate}

\noindent Our implementation works similarly, except that instead of the cubic Hermite filter, we use a two-dimensional Gauss filter.
We also use the distance $R$ as the radius of the filter, and calculate a small segment of a straight line in order to determine
how the brightness of the filter should be scaled to reach $1.5t$, so that we do not have to deal with differences depending on integration step size. 
More precisely, given the radius $R$ and filter diameter $D=2R+1$, we define a line footprint $A \in \mathbb{R}^{D\times D}$:
\begin{equation*}
    A_{x,y} = \begin{cases}
        1, \kern4em & x = R\\
        0, \kern4em & \text{otherwise}
    \end{cases}
\end{equation*}
We then apply our filter with $\sigma = R/3$, and use $1.5t  / A_{[R,R]}$ as our filter scale $s$, (in our case $t=1$).
Having obtained the filter scale, we can now compute the filtered image $L\ast I$ as $L\ast I = s \cdot Gauss(I, \sigma, R)$.
The computation of $E(I)$ is otherwise identical.

\subsection{Randomized Optimizations}
Turk and Banks define six actions:
\begin{itemize}
    \item \textbf{Insert, Delete:} Add or remove a line from the image.
    \item \textbf{Lengthen, Shorten:} In-/Decrease the length of a line on one or both ends.
    \item \textbf{Combine:} Join two lines head-to-tail.
    \item \textbf{Move:} Translate the seed of a line by a small distance.
\end{itemize}
These actions are selected randomly with random parameters, then applied to a random line.
The algorithm terminates after an energy range was reached, or accepted changes become rare enough to not introduce changes anymore.\\
\begin{minipage}{.55\textwidth}
    \vspace{5pt}
    If the change was deemed beneficial according to a decrease in energy, it is accepted, otherwise the changes are reverted.
    This causes a "drift" of the lines toward a more uniform energy level.
    Naturally, this depends heavily on the choice of $t$. If $t$ were to be chosen closer to 2,
    the image would become very crowded to reach the increased target gray level.
    \vspace{5pt}
\end{minipage}
\begin{minipage}{.45\textwidth}
    \begin{center}
        \includegraphics*{figures/SL_bump1.png}
        \captionof{figure}{A line (green) being shifted away from an existing one (black)}
    \end{center}
    \vspace*{5pt}
\end{minipage}
We have chosen to keep most of these actions as-is, the only difference introduced is
a change to how the lengthening and shortening is done. 
Instead of the two binary choices of lengthen/shorten and front/back, which only add/subtract a tiny bit at a time,
we decided to choose a segment count at random between -5 and 5 for each end.
This allows faster growth/shrinking (and hence faster convergence) while still preventing overlaps.

\subsection{Initial Seeding}
We prepare the image for the optimization routine by adding many streamlets with seeds on a regular grid to the image.
This can also be done randomly yielding similar image quality,
however strided access is more efficient with little to no benefit for the latter.

\subsection{Oracle}
The oracle from Turk and Bank's algorithm is used to suggest shorten/lengthen and move operations.
Our oracle focuses on shorten/lengthen suggestions only.

\subsection{Adding time coherence}
We added two important modifications to the aforementioned algorithm to make it partially time-coherent.
The first modification affects how seeds are chosen in the beginning of an optimization pass; the second affects
how the energy measure is computed and lines are guided toward their final positions.

\subsubsection*{Shattering}
At the end of a time step's optimization phase, we break every streamline apart into smaller streamlets.
This leaves each line with the appearance of simply being a dashed line, with each fragment having its own seed.
The seeds obtained this way are then used as the initial seeding strategy for the subsequent frame; the regular grid is only used for the first frame.
This way, we obtain many seeds that, if the field does not change too much, will quickly merge back into the line they came from.
If the field \textit{does} change, some segments will still reconnect and therefore keep their temporal coherence,
whereas areas of strong fluctuation will connect to different seeds.
This results in changes being limited to parts where change is necessary, and not affecting streamline trajectory too much on a global level.

\subsubsection*{Coaxing}
Since the energy function is used to move the image toward a constant desired target brightness, we achieve a uniform spacing of lines.
Unfortunately, this does not guarantee the lines be placed at similar positions as they were in the previous frame.
In order to coax the algorithm into favoring previous line positions, we modify $t$ to not be a constant anymore.
Instead, given the previous frame's $L$ (written $L'$) we replace $t$ with $T\in\mathbb{R}^{x\times y}$ defined as 
\begin{equation*}
    T_{x,y} = t + \left(L'_{x,y}\bigg|_0^{2t} - t\right) \cdot 0.4
\end{equation*}
The remaining operations stay the same, so our new energy expression becomes:
\[E(I, T) = \int_x\int_y\left[(L\ast I)(x,y)-T(x,y)\right]^2\,dx\,dy\]
This gives us some more fine-grained control of where lines will end up.
Due to the squishing of the typical $[0, \approx2t]$ domain of the last filtered image to only $[-.4t, .4t]$,
we soften the impact of the previous frame.
Otherwise, adding a line at the exact same position would only yield the target brightness,
causing a bunching of lines around these footprints.\\
How does choosing the Gauss filter impact the ditches left by time coherence in the field? 

\subsubsection*{Combined}
\begin{minipage}{.55\textwidth}
    Combining shattering and coaxing, we obtain a somewhat reliable way of generating streamlines according to the footprint left behind by the last frame.
    The seeds created during the shatter process all lie inside the "valley" left behind by the previous streamline path.
    Due to the coaxing function of the modified energy measure,
    it is unlikely that they will leave this valley without a change in the field forcing them to.
\end{minipage}
\begin{minipage}{.45\textwidth}
    \includegraphics{figures/SL_bump2.png}
    \captionof{figure}{The Fragments' seeds (magenta) "trapped" inside the footprint of previous line}
\end{minipage}
Due to the seeds being held in place in this way, it is very likely for them to re-join to form the same lane they originated from.
If the field changes drastically in this region, the seeds can not fully connect to each other anymore, and will instead gravitate to a different footprint,
forming long patches of coherent lines with interconnections between different paths.

  %%%%%%%%%%%%%%%%%%%%%%%%%%%%%%%%%%%%%%%%%%%%%%%%%%%%%%%%%%%%%%%%%%%%%%%%
\chapter{Implementation}
\label[section]{sec:implementation}
%%%%%%%%%%%%%%%%%%%%%%%%%%%%%%%%%%%%%%%%%%%%%%%%%%%%%%%%%%%%%%%%%%%%%%%%
This chapter briefly mentions the used libraries, and focuses on the initial implementation of - and iterative additions to - the proposed algorithm.
\section{Libraries}
The algorithm is implemented in Python3.10, and heavily relies on three libraries which are not part of the Python3.10 standard library:
\begin{itemize}
    \item ParaView v.5.12.0: A Scientific visualization software, combining data science and interactive visualization while providing custom algorithm support via the VtkPythonAlgorithm base class.
    \item VTK v.9.3.20231030 : The library used to manage anything related with the data to be visualized in ParaView.
    \item NumPy v.1.23.4: Widespread data manipulation/scientific computing library, which is used to edit the data encapsulated by VTK's objects.
\end{itemize}

\section{Time Coherent Algorithm Implementation}
Using the ``Visualization Tool Kit (VTK)'' and the ``Parallel Viewer (ParaView)'' provides a broad spectrum of available algorithms.
The most important components used are briefly listed in the following section,
the final algorithm design is delivered afterwards in \Cref{sec:impdesign},
and we conclude this chapter with a complexity analysis in \Cref{sec:impcomplexity}.

\subsection{Vital External Software Components}
\label[section]{sec:impvital}
\paragraph*{VTK Pipeline} While an extremely involved topic with dozens of hours to be spent on reading,
we try to summarize it as follows:
The VTK pipeline consists of three types of objects: Sources, filters, and sinks. Sources create data, filters modify it, and sinks display it.
Each object has input and output ports, how many of which it has decides its membership of one of the above groups.
A simple workflow to visualize a vector field would be: Vector Field Source $\rightarrow$ Line Geometry Generation $\rightarrow$ Screen Rendering.
\paragraph*{vtkPythonAlgorithm} This class is intended as a base class for writing custom algorithms in either of the three categories.
We use this twice: Once for a group of vector field sources to test our implementation with, and for the algorithm itself.
In the former case, we use it as a source, in the latter as a filter. The relevant method in this class is called "RequestData", which is passed an information object.
We can modify this object in order to pass data forward (and backward, though we do not need this) through the pipeline.
\paragraph*{vtkImageData} Objects of type vtkImageData hold a grid defined by 2 3-vectors: The extents (number of points in each direction), and the spacing (how far points are apart in X/Y/Z direction).
Every point on this rectilinear grid can have scalars or other objects assigned, like a velocity as a 3D vector.
In our case, we use it exactly in this way:
Points are assigned velocities, which are then interpolated as needed.
\paragraph*{vtkStreamTracer} The vtkStreamTracer class is a filter with two inputs: We provide it with a vector field (vtkImageData) object and a point, which it then integrates through the field.
The relevant output for us is the list of points making up the streamline that it returns.
\newpage
\subsection{Algorithm Design}
\label[section]{sec:impdesign}
Since we need two filters ($L_s, L_t$), and want them to act on different time steps, we have decided to implement the filtering and drawing subsystem as follows:
\paragraph*{FilterTarget} Effectively a wrapper for an image, allowing easier access to some properties.
It contains the brightness information of the image that guides our algorithm.
\paragraph*{Painter} This is the modifying actor for FilterTargets.
Painters use a configuration (line brightness, blur size, etc.)
and draw poly lines to the FilterTargets accordingly. This is how we distinguish between $L_t$'s and $L_s$'s radii.
\paragraph*{Filter} These objects contain a list of lines that make up the vector field image,
and orchestrate the assigned Painters and their respective configuration objects.
They also provide methods for adding/removing/modifying lines,
using a given energy function to determine their success.
They act as the binding agent for the logic modifying line placement and actually performing the change.
\paragraph*{FilterStack} This class is best used (though not enforced) as a singleton;
it manages two filter objects, one for the current,  and one for the last time step.
It also provides the energy methods as lambdas to the new filter added every time step.

This change compared to the original was necessary,
because we now want to manage multiple filters from different time frames at the same time.
It is even possible to make the filter change depending on how long ago it was created,
e.g. to not only use time coherence w.r.t the last frame,
or to allow effects like onion skinning of older frames' low pass images.

The entry point for this algorithm is, as with any vtkPythonAlgorithm,
the ``RequestData'' method (we leave out the other Request() methods for brevity).
We are provided the vector field via the vtkImageData object as input,
and start to set up our low-pass filter stack.

By setting up a filter with a config
(the standard config uses similar values as Turk and Banks' implementation), we create the $E_s$ part.
If we are not interested in time coherence,
this is all that is necessary for a line to be drawn filter-wise.
Otherwise, we simply add another config specialized to work well with $E_t$,
so our filter now has two configs, painters, and targets: One for $E_s$, one for $E_t$.

Drawing the lines themselves is done using NumPy's vectorization,
since we can use the NumPy-compatible vtkDataSetAdapter (DSA).
We use this to quickly obtain and transform the coordinates returned from the vktStreamTracer.
The drawing process is handled entirely by the Painter objects:
For a line $L$ containing $n$ seeds, we calculate the bounding boxes of $n-1$ segments,
padded by the filter radius.
Each pixel inside this rectangle has a number of vectorized calculations
performed in order to determine its brightness.
The brightness is evaluated using a precomputed grayscale table which we
interpolate via SciPy's RegularGridInterpolator, as this also supports vectorized access.
Once each segment's pixels are computed, we simply add them to the global line image.

Having finished the drawing process, we now look at the energy measure.
The filter stack hands over a lambda to the respective filter, with some arguments bound to their respective FilterTarget.
This way, we can dynamically change how the filter calculates values based on the gray scale values form the bound targets.
If, e.g., we do not have an old filter yet, we cannot use the \textit{coaxing} strategy.
Therefore, we simply leave the argument bound to "None" when passing it to the first filter.


\subsection{Complexity Analysis}
\label[section]{sec:impcomplexity}
\paragraph*{Line Integration}
We heavily rely on the vtkStreamTracer (see \Cref{sec:impvital}),
which we configured to use the Runge-Kutta-4 solver.
RK4 uses a complexity of $O(n)$, with $n$ being the number of integration steps taken.
Hence, obtaining the sample points of a single streamline of length $n$ is of complexity $O(n)$.

\paragraph*{Drawing a Line Segment}
When computing the footprint of a streamline, we do so segment by segment.
A segment footprint's pixel count $f$ is defined by two values:
The filter radius $r$ and the maximum step length $l$.
The maximum cost of drawing a segment is therefore a constant, and we write this cost as $c$.

\paragraph*{Drawing a Line}
If the line consists of $n$ segments each costing a maximum of $c$, the cost is simply $O(n)$.
This remains unchanged when including the cost of integration, as we still get $O(2n) = O(n)$.

\paragraph*{Generating a Streamline}
Lines start at very low length, and need to grow and shift, causing them to be redrawn over and over.
Generating a single line by letting it grow means we re-evaluate it as often as necessary.
We need to draw $O(0.5\cdot n^2)=O(n^2)$ segments to reach a length of $n$,
combining this with the cost of drawing a length-$n$ line, we arrive at $O(cn^2)=O(n^2)$.
\\\\
The randomized operations all cause a line to be re-generated, but since its length remains, this is done
with $O(n)$. However, we do this fairly often, making the complexity w.r.t. segment size and iteration count
behave as if it were quadratic.

  %!TEX root = ../Thesis.tex

%%%%%%%%%%%%%%%%%%%%%%%%%%%%%%%%%%%%%%%%%%%%%%%%%%%%%%%%%%%%%%%%%%%%%%%%
\chapter{Results}
\label{chap:Results}


\begin{figure}[t]
  \subfloat[\label{fig:one}]{%
    \exampleDuck[width=0.45\linewidth]%
  }%
  \hfill%
  \subfloat[\label{fig:two}]{%
    \exampleDuck[width=0.45\linewidth]%
  }%
  \caption{Example for two~\protect\subref{fig:one} sub-figures~\protect\subref{fig:two}.}
\end{figure}

In \autoref{chap:Results}, \autoref{sec:section}, we will see bla, specifically in \autoref{sec:subsection} this will be emphasized. \Blindtext[2]
%%%%%%%%%%%%%%%%%%%%%%%%%%%%%%%%%%%%%%%%%%%%%%%%%%%%%%%%%%%%%%%%%%%%%%%%
\section{Section}
\label{sec:section}

\Blindtext[6]
%%%%%%%%%%%%%%%%%%%%%%%%%%%%%%%%%%%%%%%%%%%%%%%%%%%%%%%%%%%%%%%%%%%%%%%%
\subsection{Subsection}
\label{sec:subsection}

\Blindtext[3]
%%%%%%%%%%%%%%%%%%%%%%%%%%%%%%%%%%%%%%%%%%%%%%%%%%%%%%%%%%%%%%%%%%%%%%%%
  %!TEX root = ../Thesis.tex

%%%%%%%%%%%%%%%%%%%%%%%%%%%%%%%%%%%%%%%%%%%%%%%%%%%%%%%%%%%%%%%%%%%%%%%%
\chapter{Conclusion}

\Blindtext[16]
%%%%%%%%%%%%%%%%%%%%%%%%%%%%%%%%%%%%%%%%%%%%%%%%%%%%%%%%%%%%%%%%%%%%%%%%  

% This ensures that the subsequent sections are being included as root
% items in the bookmark structure of your PDF reader.
\bookmarksetup{startatroot} 
\backmatter
  
  \printbibliography
  
\end{document}