%!TEX root = ../Thesis.tex

\begin{center}
  \textsc{Abstract}
\end{center}
% Grammarly'd
\noindent
Common elements in visualizing steady vector fields are integral lines that show flow dynamics, called streamlines.
Many possible criteria can define the quality of a set of streamlines when visualizing a vector field.
Frequently used placement properties include the uniformity and length of the streamlines, with even spacing and maximum length being the preferred choice for high visual appeal.
We have noticed that the most relevant criterion for visualizing $unsteady$ fields using e.g.,
animations, is largely unaccounted for.
Currently, most algorithms can only animate a vector field frame-by-frame without time-awareness.
The result is many visually pleasing images lacking coherence between each other,
as said algorithms place streamlines merely to fulfill the frame-limited optimum.
We overcome this problem by introducing an algorithm capable of time-coherent streamline placement,
effectively minimizing streamline movement between frames.
We implement an image-guided approach based on the work by Greg Turk and David Banks \cite{TurkBanks} to
generate initial streamline placements according to the above criteria.
To add time coherence, several key changes to different components are made.
The first change revolves around the energy measure, a function defining
the uniformity of streamline placement by comparing it with a target brightness.
Another change is the kernel Turk and Banks use to blur the image,
which we adapt to include time coherence, as well as introducing a new seeding strategy.
We show how the individual changes affect the streamline placement locally and globally,
and apply the algorithm to various datasets.
\cleardoublepage
