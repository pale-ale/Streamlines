%!TEX root = ../Thesis.tex

\begin{center}
  \textsc{Abstract}
\end{center}
%
\noindent
A common element used to represent steady vector fields are integral lines capable of showing flow dynamics, called streamlines.
There are many possible criteria that can be used to define the quality of a set of streamlines when visualizing a vector field.
Commonly used placement properties include the uniformity and length of the streamlines, with even spacing and maximum length being the preferred choice for high visual appeal.

We have noticed that an important criterion for creating visualizations of $unsteady$ fields, such as animations, is largely unaccounted for.
Currently, algorithms can only animate a vector field frame-by-frame without actually being time-aware.
The result is many visually pleasing images, however there is no coherence between each image, as lines are only placed to fulfill the frame-limited optimum.

In this work, an algorithm capable of time coherent streamline placement---that is lines that move as little as possible between frames--- is introduced.
An image-guided approach based on the work by Turk and Banks will be implemented to generate initial streamline placements according to the above criteria.
We then introduce modifications to several components.
The first key change revolves around the energy measure, a function used to define the uniformity of line placement by comparing it with a target brightness.
Another change is the kernel Turk and Banks use in order to blur the image, which we also adapt for time coherence in combination with a new seeding strategy.
We show how the individual changes affect the streamline placement locally and globally, and apply the algorithm to various datasets.
\cleardoublepage
