%!TEX root = ../Thesis.tex

\begin{center}
  \textsc{Zusammenfassung}
\end{center}
%
\selectlanguage{ngerman}
Ziel dieser Arbeit ist das Erreichen von Zeitkohärenz bei Stromlinien
in zeitabhängigen Vektorfeldern, um deren Darstellung durch z.B. Animationen zu verbessern.
Wir beginnen mit einer Definition für Zeitkohärenz, die im Wesentlichen auf die
Bewegung einer Stromlinie von einem Zeitschritt zum nächsten gestützt ist.
Wenn diese Bewegung groß ausfällt bzw. sich die Stromlinie viel bewegt hat,
bezeichnen wir dies als ``schlechte Zeitkohärenz''.
Wir zeigen, warum zeitkohärentes Verhalten wichtig ist für z.B. Animationen,
und entwickeln eine prototypische Implementierung, deren Parametrisierung
wir dann anhand einiger Beispiele bestimmen.
Unser Algorithmus basiert auf einer durch Bild-geleiteten Implementierung von Greg Turk und David Banks,
den wir durch einige Änderungen an den zentralen Komponenten zeitkohärent gemacht haben.
Im Anschluss werden einige Datensätze damit ausgewertet und mit anderen einfachen Algorithmen verglichen;
außerdem beschreiben wir einige Limitierungen sowie Ideen für Verbesserungen.
Zum Schluss wird eine kurze Komplexitätsanalyse durchgeführt und die Performance untersucht.
\selectlanguage{english}
\cleardoublepage
