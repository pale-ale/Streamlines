%%%%%%%%%%%%%%%%%%%%%%%%%%%%%%%%%%%%%%%%%%%%%%%%%%%%%%%%%%%%%%%%%%%%%%%%
\chapter{Fundamentals}
%%%%%%%%%%%%%%%%%%%%%%%%%%%%%%%%%%%%%%%%%%%%%%%%%%%%%%%%%%%%%%%%%%%%%%%%
\section{Vector Fields}
A vector field represents how vectorized elements act over a spatial domain.
Intuitively, this means that for every point in a domain we can obtain the force that acts at that point.

\subsection{Definitions}
More formally, a vector field can be defined as a map from a domain of points to a vector.
We can write it as an $n$-$m$-valued function, mapping an $n$-dimensional input to an $m$-dimensional output.
In this thesis, we will only care about cases of $n=m$ in two and three dimensions.

There are several ways such fields can be obtained, a simple algebraic definition could be e.g. $u(x,y) = (1,0)$.
This gives us a field that represents a force of magnitude one towards positive $x$ and no influence on the $y$ component.

If we want our force to not only depend on spacial input, but also on another scalar like a time component, we write this as $u(x,y,t)$.

Vector fields are called \textit{steady} if they do not have a time component, otherwise they are referred to as \textit{unsteady}.
Another distinction is \textit{continuity}, this is the same as the algebraic definition for other functions.
The fields in this paper are all going to be continuous.

\subsection{Critical Points}
A vector field can have points with special characteristics, called critical points.
In the 2D case, there are only four commonly used critical points:
\begin{itemize}
    \item \b{Source}:
    Given a field such as $u(x,y) = (x,y)$, we can see that to every point applies a force away from the origin.
    If we think about this as the flow of a liquid, then this would mean that (in the case of noncompressible flow) liquid is \textit{created} at the point $(0,0)$.
    We therefore refer to such a point as a \textit{source}.
    \item \b{Sink}:
    Similarly, $u(x,y) = (-x,-y)$ would give us a \textit{sink} at $(0,0)$, essentially destroying liquid.
    \item \b{Saddle}:
    A saddle is an area where liquid is compressed in one direction, and stretched in another, e.g. by $u(x,y) = (-x,y)$
    \item \b{Periodic orbit}:
    $u(x,y) = (-y,x)$ creates circular paths around the origin, where after travelling a distance of $2\pi r$ you will end up at the point you started with.
    These critical points are therefore called \textit{periodic orbits}.
\end{itemize}

\section{Streamlines}
Given a vector field $u$ and a point $P$, we can trace the movement of this point through $u$ by integrating over the field.
Intuitively, we just step through the field by choosing the next point $P_n = P_{n-1} + c * u(P_{n-1})$, with $c$ being a step size scale.
If we do this an infinte number of times with positive and negative values for $c$,
we end up with a set of points $S$ we have passed through, which defines the streamline.
$S$ has two notable properties:
\begin{itemize}
    \item For every point $P$ inside this set, the direction of its derivative is equal to $u(P)$.
    This means that the streamline is tangent to the vector field at every points.
    \item No matter which point inside $S$ we use as $P_0$, we will always obtain the same set $S$ as its streamline.
\end{itemize}
Any point inside $S$ is referred to as a \textit{seed}, yielding the streamline $S$.

\subsection{Spatial Coherence}
If we want to visualize a vector field, we want its features to be easily identifiable.
At the same time, we do not want to introduce distractions or artifacts due to the visualization technique.
Deciding factors of uniformity in streamline visualization are streamline length and density.
Longer streamlines make for a smoother appearance, whereas many short lines tend to obfuscate and hinder recognition of important features like critical points.
\subsection{Maybe Temporal Coherence?}