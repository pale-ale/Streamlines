%!TEX root = ../Thesis.tex

%%%%%%%%%%%%%%%%%%%%%%%%%%%%%%%%%%%%%%%%%%%%%%%%%%%%%%%%%%%%%%%%%%%%%%%%
\chapter{Results}
\label{chap:Results}
In this chapter, we start by running the full algorithm on various datasets while
discussing the different outcomes depending on the coherence strategy, kernels, and coherence weights.
Then, we show some performance metrics and the hardware configuration.
We finish this chapter with problems and limitations regarding the functionality
and choice of parameters of our algorithm.

\section{Visualizing different fields}
The comparisons for the various fields are made between Turk and Banks' algorithm without coherence,
a slightly modified approach simply reusing the line's seeds for successive frames, and our modified version
using coaxing and shattering enabled.
We start this section with three simple fields.
\[(a): u(x,y) = (1,0) \hspace{7mm} (b): u(x,y,t)=(1,t) \hspace{7mm} (c): u(x,y,t)=(x-t, y-t)\]
Next, we use the algebraically defined fields from previous parts of this thesis.\\

\noindent Finally, a dataset of size 60 x 30 is used.

\section{Performance}
The algorithm was run on a system with the following relevant hardware.
CPU: AMD Ryzen 5 5600X 3.70 GHz, RAM: 16 GB DDR4 ($\approx$ 400 MB used by ParaView).
Most of the images with a black background color were generated in about 3-4 minutes each.
Image ??? from the hot room dataset took ??? h m s.
For comparison, the initial greedy algorithm took about 10 seconds.
The size of the low pass image is of little no importance for speed, we tested filter radii between 2 and 32,
both of which took about the same time to complete.
\textit{Some better numbers/tables}

\section{Issues and Limitations}
Strong changes lead to drastic degradation of image quality when using high coherence weights.
This can be counteracted by either interpolating the field movement, or reducing $\alpha$ to allow the lines to move more freely.
Memory usage scales with the low pass size and line count, as every line saves its contributions to the energy as an array of the same size.
When using large amounts of lines with a high low-pass resolution, memory usage can increase drastically.
We found that the low-pass resolution has very little effect on the resulting line placement, we used the 120x120 resolution solely for a smoother appearance
The coherence weight parameter is configurable prior to a run,
but for different datasets (more specifically: different time strides) it is often useful to vary this parameter.
This could perhaps be added via an adaptive method,
generating different images from one previous image using different weights,
and selecting the one with the least total energy.
Of course, this would increase the computational cost drastically,
and we have therefore decided not to add this feature for now.

%%%%%%%%%%%%%%%%%%%%%%%%%%%%%%%%%%%%%%%%%%%%%%%%%%%%%%%%%%%%%%%%%%%%%%%%