%%%%%%%%%%%%%%%%%%%%%%%%%%%%%%%%%%%%%%%%%%%%%%%%%%%%%%%%%%%%%%%%%%%%%%%%
\chapter{Related Work}
\label[section]{sec:relatedWork}
%%%%%%%%%%%%%%%%%%%%%%%%%%%%%%%%%%%%%%%%%%%%%%%%%%%%%%%%%%%%%%%%%%%%%%%%

Most of the work published in the field of streamline placement algorithms can be divided into three categories,
each having different focuses, strengths, and drawbacks.
Each category will be briefly addressed in the following sections.

\section{Image-Guided Algorithms}
The goal of this approach is to achieve a very uniform spacing of streamlines,
giving an almost "hand-drawn" appearance.
Generated images will usually have high visual quality,
at the risk of potentially missing or misrepresenting features and higher computational cost.

One of the first and most prominent examples in this category was created
by Greg Turk and David Banks~\cite{TurkBanks} in 1996,
which we have adopted as the base for our algorithm as well.
In their research paper, a function (called the ``Energy Function'')
is defined such that it maps an input image containing the potential streamlines to a scalar.
The scalar represents the quality of the image, roughly defined as the uniformity of the streamline spacing. 
Adding/moving/removing/resizing streamlines is done randomly,
at every step the energy function is used to determine whether a proposed change gets accepted or discarded.
The algorithm is finished when the energy function reaches a lower energy bound or the user stops it.
This idea was extended by \cite{https://doi.org/10.1111/j.1467-8659.2009.01352.x} to allow
streamline generation on 3D surfaces.
Another image-guided approach, which is also view-dependent for 3D fields, was created by
\cite{8093671}.


\section{Feature-Guided Algorithms}
Feature-guided algorithms examine the underlying field structure when placing seeds.
They search for critical points or patterns in the field and then seed around them,
capturing them in much higher detail.
The resulting images inherently represent the critical points much better,
sometimes at the cost of visual appeal compared to image-guided algorithms.
\cite{Verma} is one of the first algorithms in this category, extracting field topology
and then applying a fixed pattern of seeds for the different structures to capture them
as accurately as possible.
\cite{1532832} developed a farthest-point seeding strategy using Delaunay triangulation
for fast seed positioning.
\cite{wu2009topology} picks up on this idea, improving it to generate better images still adhering
to the underlying topology.

\section{Greedy Algorithms}
Greedy algorithms do not use a global guiding principle for streamline placement.
We used an approach similar to \cite{Jobard} as a prototype,
placing seed candidates along a generated line and attempting to grow streamlines 
from them until the space is filled.
\cite{4015453} later improved this algorithm's speed by an order of magnitude.
\cite{Mattausch} extended this principle into 3D,
adding several features to make it more suitable for user interaction and 3D rendering.





