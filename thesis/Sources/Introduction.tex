%!TEX root = ../Thesis.tex

%%%%%%%%%%%%%%%%%%%%%%%%%%%%%%%%%%%%%%%%%%%%%%%%%%%%%%%%%%%%%%%%%%%%%%%%
\chapter{Introduction}
%%%%%%%%%%%%%%%%%%%%%%%%%%%%%%%%%%%%%%%%%%%%%%%%%%%%%%%%%%%%%%%%%%%%%%%%
% Grmmarly'd
Vector field visualization is a ubiquitous component encountered in many processes used by science and industry alike,
such as aerospace engineering, geology, fluid dynamics, material science, or the biomedical sector.
Frequent use cases for vector fields include weather systems, rotor design,
analysis of marker movement inside a cell, or predicting the magnetic field of a star.
Displaying such fields in a human-focused way is invaluable when analyzing the
underlying systems' dynamics, allowing for faster and more accurate data analysis
and subsequent decision-making.
Contemporary research has a primary focus on continuous and steady vector fields.
Streamlines are usually optimized w.r.t. a single frame, allowing for e.g.,
a spatially uniform and therefore visually pleasing streamline placement.

A problem arises when small changes over time are introduced, i.e.,
an unsteady vector field is used instead of a steady one.
While current algorithms still produce good images frame by frame,
it is unlikely for these algorithms to prefer streamlines similar to the ones in the previous frame.
Due to this, streamlines move about and shift between frames a lot more than the time-induced changes to the field would warrant,
with the sporadic movement making it very difficult to perceive the flow change of the field.
Creating an animation of an unsteady vector field's behavior thus requires frequent human intervention,
which makes good animations very tedious, time-consuming, and ultimately expensive.
Most vector fields represent dynamic systems, which makes many areas susceptible to this limitation.

We believe that by introducing a new criterion that describes how much streamlines move from one frame to another,
we can quantify the visual fidelity of streamline placements generated for different frames.
In this thesis, we will refer to this criterion as \textit{time coherence}, indicating that
the objective is to constrain streamline movement from one frame to the next to ensure
that the human eye perceives the ``movement'' of a streamline as one fluid motion
through several frames.
\newpage

This paper will focus on three core aspects revolving around time coherence:
\begin{itemize}
    \item A motivation for---and definition of---time coherence by deriving
    a measure from simple cases lacking time coherence.
    \item The implementation and underlying ideas of an image-based algorithm generating
    evenly-spaced long streamlines, which we use as the basis for our implementation.
    \item Adaptations of the base algorithm to allow a controlled bias between similarity
    to a previous time step, and optimality w.r.t. the spatial layout for the current step.
    We show how time coherence can be added in a non-intrusive,
    compatible way to combine it with spatial coherence.
\end{itemize}
We have chosen an image-based approach
(as opposed to a feature-guided one, see \Cref{sec:relatedWork} for a disambiguation) for our work
because the movement of streamlines between time frames is an appearance-focused problem,
and therefore better suited by an appearance-focused algorithm.
Another reason is that feature-guided algorithms usually act locally,
whereas the movement of a line from one step to another is a global constraint.

The succeeding sections are structured as follows:
In \Cref{sec:relatedWork}, we briefly note and classify some groups of algorithms,
and describe how our algorithm fits into these surroundings.
The fundamentals, mostly in the areas of vector field visualization and image processing, are
provided in \Cref{sec:fundamentals}.
\Cref{sec:method} starts with a brief segment about a discontinued heuristic algorithm,
followed by a close examination of the image-guided algorithm we then decided to use as a foundation.
Time coherence is motivated and explained in greater detail,
with initial ideas regarding its implementation laid out, then executed in \Cref{sec:implementation}.
\Cref{sec:results} contains our findings and ideas for future work,
with \Cref{sec:conclusion} concluding this thesis.


% \paragraph*{Related Work, Chapter 2} We present some recent developments and concepts.
% \paragraph*{Fundamentals, Chapter 3} Fundamentals about vector fields and streamlines are listed and briefly explained.
% \paragraph*{Method, Chapter 4} The base algorithm is introduced alongside the definition of a
% criterion for time coherence. Changes made to the algorithm on account for time coherence are discussed.
% \paragraph*{Implementation, Chapter 5} We describe the implementation and some design ideas.
% \paragraph*{Results, Chapter 6} We compare different algorithms on several datasets, and we discuss limitations,
% algorithm performance, and its complexity.
% \paragraph*{Conclusion, Chapter 7} In the final chapter, we conclude the thesis with
% a brief summary of the presented topics and some ideas for future work.
