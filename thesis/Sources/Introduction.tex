%!TEX root = ../Thesis.tex

%%%%%%%%%%%%%%%%%%%%%%%%%%%%%%%%%%%%%%%%%%%%%%%%%%%%%%%%%%%%%%%%%%%%%%%%
\chapter{Introduction}
%%%%%%%%%%%%%%%%%%%%%%%%%%%%%%%%%%%%%%%%%%%%%%%%%%%%%%%%%%%%%%%%%%%%%%%%

The visualization of vector fields is a heavily used component in many fields like aerospace engineering, geology, fluid dynamics, material science, or the biomedical sector.
Common examples include weather systems, rotor design, analysis of marker movement inside of cells, or the magnetic field of a star.
Displaying such fields in a human-focused way is invaluable when analyzing the underlying systems' dynamics, and allows for faster and more accurate data analysis.

Contemporary research is mostly focused on continuous and steady vector fields.
Lines are usually optimized with respect to a single frame, which in turn has a very pleasing appearance.
If we introduce small changes over time, i.e. use an unsteady vector field, it is unlikely those algorithms will prefer lines similar to the ones in the last frame.
This causes the streamlines to move about and shift between frames a lot more than the time-induced changes to the field would warrant.
The sporadic movement makes it very difficult to perceive the flow change of the field.
Therefore, creating an animation of an unsteady vector field's behavior is very difficult,
time-consuming, and requires a lot of human intervention in order to achieve good results.
This is extremely limiting, as most (if not all) vector fields represent a dynamic system.
Therefore, the focus of this thesis will be threefold:
\begin{itemize}
    \item How a lack of coherence of streamlines in different time steps of a continuous, unsteady field affects the visual quality of a field.
    A criterion to measure time coherence will also be presented.
    \item The implementation and underlying ideas of an image-based algorithm generating evenly-spaced long streamlines.
    \item Adaptations of the algorithm to allow a controlled bias between similarity to a previous time step, and optimality w.r.t. the current step.
\end{itemize}
We have chosen an image-based approach (opposed to a feature-guided one, see Section 2 for a disambiguation) for our work,
because the movement of lines in-between time frames is better suited for an appearance-focused algorithm.
Another reason is that feature-guided algorithms usually act locally, whereas the movement of a line from one step to another
is a global constraint.
The succeeding sections are structured as follows:
\begin{description}
    \item [Chapter 2] presents some recent developments and concepts.
    \item [Chapter 3] contains fundamentals about vector fields and streamlines.
    \item [Chapter 4] is about the base algorithm, the definition of a criterion for time coherence, and changes made to account for time coherence.
    \item [Chapter 5] contains the implementation and some design ideas.
    \item [Chapter 6] examines the individual changes' effects on different datasets, and we discuss issues, performance, and other measurements.
    \item [Chapter 7] concludes the thesis with some ideas for future work and improvements.
\end{description}
