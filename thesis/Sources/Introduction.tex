%!TEX root = ../Thesis.tex

%%%%%%%%%%%%%%%%%%%%%%%%%%%%%%%%%%%%%%%%%%%%%%%%%%%%%%%%%%%%%%%%%%%%%%%%
\chapter{Introduction}
%%%%%%%%%%%%%%%%%%%%%%%%%%%%%%%%%%%%%%%%%%%%%%%%%%%%%%%%%%%%%%%%%%%%%%%%

The visualization of vector fields is a heavily used component in many fields like aerospace engineering, geology, fluid dynamics, material science, or the biomedical sector.
They play an important role in many design processes, and facilitate the understanding of material dynamics.
Common examples are weather systems, rotor design, analysis of marker movement inside of cells, or the magnetic field of a star.
Displaying such fields in a palpable way is an important part in understanding the underlying dynamics and allows for better analysis of the data.

Contemporary research is mostly limited to continuous and steady vector fields, making animation of unsteady fields very difficult.
Lines are usually optimized with respect to a single frame, which in turn has a very pleasing appearance.
If we introduce and small, localized changes to the field, it is unlikely that similar lines will be chosen or preferred.
This causes lines to move about and shift between frames a lot more than the time-induced changes to the field would warrant.
The sporadic movement makes it very difficult to perceive the flow change of the field.
Therefore, creating an animation of an unsteady vector field's behavior is very difficult,
time-consuming, and requires a lot of human intervention in order to achieve good results.
This is extremely limiting, as most (if not all) vector fields represent a dynamic system.

Therefore, the focus of this thesis will be on three main topics:
\begin{itemize}
    \item How a lack of coherence of streamlines in different time steps of a continuous, unsteady field affects the visual quality of a field.
    A criterion to measure time coherence will also be presented.
    \item The implementation and underlying ideas of an image-based algorithm generating evenly-spaced long streamlines.
    \item Adaptations of the algorithm, to allow a controlled bias between similarity to a previous time step, and optimality w.r.t. the current step.
\end{itemize}

We have chosen an image-based approach (opposed to a feature-guided one, see Section 2 for a disambiguation) to our work,
because the movement of lines in-between time frames is better suited for an appearance-focused algorithm.
Another reason is that feature-guided algorithms usually act locally, whereas the movement of a line from one step to another
is a global constraint.

The succeeding sections are structured as follows:
\begin{itemize}
    \item Section 3 contains some fundamentals about vector fields and streamlines
    \item Section 4 will be about the definition of the criterion of time coherence
    \item The initial algorithm is examined in section 5
    \item In Section 6, we introduce changes to the algorithm and examine their individual effects on different datasets
    \item The implementation is more closely discussed in Section 7.
\end{itemize}
As a conclusion, we show some performance metrics, optimization issues, and some possibilities for further work.
