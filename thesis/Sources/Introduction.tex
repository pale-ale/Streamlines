%!TEX root = ../Thesis.tex

%%%%%%%%%%%%%%%%%%%%%%%%%%%%%%%%%%%%%%%%%%%%%%%%%%%%%%%%%%%%%%%%%%%%%%%%
\chapter{Introduction}
%%%%%%%%%%%%%%%%%%%%%%%%%%%%%%%%%%%%%%%%%%%%%%%%%%%%%%%%%%%%%%%%%%%%%%%%

For steady fields, there are several papers showing different strategies to generate streamlines according to different criteria.
Commonly preferred attributes are high streamline length, and uniform line density, which are often labelled as "coherence".
Both of which greatly enhance the information uptake by preventing visual clutter;
thereby allowing a focus on more important features and characteristics of the field at hand.

\section{Problem Statement}
As soon as a change in the field is introduced, however, these methods are no longer optimal when examined over the entire time span.
This is a direct consequence of the streamline generation technique only giving optimal placement w.r.t a single frame.
If we introduce multiple time-frames and small, localized changes to the field, it is unlikely that similar lines will be chosen or preferred.
This causes lines to move about and shift between frames a lot more than the time-induced changes would warrant.
The sporadic movement makes it very difficult to perceive the way the lines would actually move due to the field changing.
Therefore, creating an animation of an unsteady vector field's behavior is very difficult, time-consuming, and requires a lot of human intervention.

\section{Proposed Solution}
We define a criterion called "time coherence", mainly referring to how streamlines move on a frame-by-frame basis. 
For a more detailed explanation of the approach, see the next section.

Since this is mostly a problem of visual fidelity, we choose an image-based streamline generation algorithm,
which we then extend to include the time coherence constraint. 
By extending it in this way, animated streamline visualizations look much better and introduce a lot less artifacts.




