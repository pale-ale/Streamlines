%!TEX root = ../Thesis.tex

%%%%%%%%%%%%%%%%%%%%%%%%%%%%%%%%%%%%%%%%%%%%%%%%%%%%%%%%%%%%%%%%%%%%%%%%
\chapter{Conclusion}
\label[section]{sec:conclusion}

We have shown a motivation for and definition of time coherence,
and discussed its importance when animating vector fields.
Two basic algorithms capable of spatial coherence were presented,
one of which we adapted to become capable of time-coherent streamline placement.
Examples for different configurations and use cases were provided and compared,
with conclusions regarding optimal parameter choices.

By running our algorithm on a multitude of different vector fields and datasets,
we have shown that it drastically improves the time coherence of images generated for
subsequent time steps in unsteady fields, and that it can be used to animate them effectively.

Some ideas for future work include choosing different base algorithms and approaches,
especially regarding performance to allow for interactive use were presented.
An implementation for 3D vector fields in combination with view-dependent line
choice may also prove to be quite interesting and useful with many areas of application.

%%%%%%%%%%%%%%%%%%%%%%%%%%%%%%%%%%%%%%%%%%%%%%%%%%%%%%%%%%%%%%%%%%%%%%%%