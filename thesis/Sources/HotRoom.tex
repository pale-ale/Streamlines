\subsection{Hot Room Dataset}

The final vector field we examine contains 60x30x60 data points.
Since the dataset is 3D and our algorithm only works in 2D,
we slice the dataset parallel to the $X,Y$-plane at half height with a $Z$-index of 30.
Because the field is more chaotic than the algebraic cases we used until now,
we decided on an iteration limit of 5000 steps compared to 1500 for the other images.
The comparison is made between our implementation and the base algorithm only,
as the constant algorithm does not produce useful images beyond the first, and we will show several.

\begin{figure}[t]
    \centering
    \begin{subfigure}{\textwidth}
        \centering
        \begin{subfigure}{.49\textwidth}
            \centering
            \includegraphics[scale=.065]{figures/Results/HotRoom/450/HotRoom.png}
            \caption{}
        \end{subfigure}
        \begin{subfigure}{.49\textwidth}
            \centering
            \includegraphics[scale=.065]{figures/Results/HotRoom/450/HotRoomL.png}
            \caption{}
        \end{subfigure}
    \end{subfigure}
    \begin{subfigure}{\textwidth}
        \centering
        \begin{subfigure}{.49\textwidth}
            \centering
            \includegraphics[scale=.065]{figures/Results/HotRoom/450/HotRoomTB1O.png}
            \caption{}
        \end{subfigure}
        \begin{subfigure}{.49\textwidth}
            \centering
            \includegraphics[scale=.065]{figures/Results/HotRoom/450/HotRoomOurs1O.png}
            \caption{}
        \end{subfigure}
    \end{subfigure}
    \begin{subfigure}{\textwidth}
        \centering
        \begin{subfigure}{.49\textwidth}
            \centering
            \includegraphics[scale=.065]{figures/Results/HotRoom/450/HotRoomTB2O.png}
            \caption{}
        \end{subfigure}
        \begin{subfigure}{.49\textwidth}
            \centering
            \includegraphics[scale=.065]{figures/Results/HotRoom/450/HotRoomOurs2O.png}
            \caption{}
        \end{subfigure}
    \end{subfigure}
    \caption{A comparison to how similar each algorithm's streamline image is compared to its previous one.
        (a, b) are the initial frame's footprint (left) and streamline placement (right) for both algorithms at time step 450.
        (c) shows the base algorithm's streamline placement (black) at step 451
        compared to (b)'s streamlines (red).
        (e) compares step 452 (black), to 451 (red).
        (d, f) work similarly with our implementation.
    }
    \label[figure]{fig:reshotroom450}
\end{figure}


We start with the dataset at timestamp 450, the individual changes shown in \Cref{fig:reshotroom450}.
It is easy to see the general trend of our implementation and the base algorithm producing images of
similar quality w.r.t. spatial placement continuing, with our version creating images exhibiting
much better time coherence.

On the next page, we evaluate the behavior of our implementation when
we skip a large amount of frames.
We start at frame 700 and generate the subsequent frame,
jump to 1000, and generate the next frame again, comparing the results.
\newpage

\begin{figure}[ht!]
    \centering
    \begin{subfigure}{\textwidth}
        \centering
        \begin{subfigure}{.49\textwidth}
            \centering
            \includegraphics[scale=.065]{figures/Results/HotRoom/700/HotRoom.png}
            \caption{}
        \end{subfigure}
        \begin{subfigure}{.49\textwidth}
            \centering
            \includegraphics[scale=.065]{figures/Results/HotRoom/700/HotRoomL.png}
            \caption{}
        \end{subfigure}
    \end{subfigure}
    \begin{subfigure}{\textwidth}
        \centering
        \begin{subfigure}{.49\textwidth}
            \centering
            \includegraphics[scale=.065]{figures/Results/HotRoom/700/HotRoomTB1O.png}
            \caption{}
        \end{subfigure}
        \begin{subfigure}{.49\textwidth}
            \centering
            \includegraphics[scale=.065]{figures/Results/HotRoom/700/HotRoomOurs1O.png}
            \caption{}
        \end{subfigure}
    \end{subfigure}
    \begin{subfigure}{\textwidth}
        \centering
        \begin{subfigure}{.49\textwidth}
            \centering
            \includegraphics[scale=.065]{figures/Results/HotRoom/1000/TB1O.png}
            \caption{}
        \end{subfigure}
        \begin{subfigure}{.49\textwidth}
            \centering
            \includegraphics[scale=.065]{figures/Results/HotRoom/1000/Ours1O.png}
            \caption{}
        \end{subfigure}
    \end{subfigure}
    \begin{subfigure}{\textwidth}
        \centering
        \begin{subfigure}{.49\textwidth}
            \centering
            \includegraphics[scale=.065]{figures/Results/HotRoom/1000/TB2O.png}
            \caption{}
        \end{subfigure}
        \begin{subfigure}{.49\textwidth}
            \centering
            \includegraphics[scale=.065]{figures/Results/HotRoom/1000/Ours2O.png}
            \caption{}
        \end{subfigure}
    \end{subfigure}
    \caption{
        Same image layout as \Cref{fig:reshotroom450}, with a 300 time step jump
        between the top and bottom four images.
    }
    \label[figure]{fig:resfinal}
\end{figure}

We see that the jump in time steps occurred between \Cref{fig:resfinal} (c) and (e), and (d) and (f),
as the currently shown streamline after image does not match the afterimage at all.
Even for this case, our algorithm creates lines adhering well to the spatial criteria,
with time coherence not achieved in most regions, as is to be expected since most of the field is vastly
different due to the large time step difference.
The algorithm quickly recovers from this discrepancy,
as the next frame has regained most of its time coherence.
