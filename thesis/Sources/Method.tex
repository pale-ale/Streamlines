\chapter{Method}
\label[section]{sec:method}
In this chapter, we describe the motivation and concepts used for developing
an algorithm supporting time coherence.
We start with an implementation we ultimately deemed infeasible in \Cref{sec:failedalg},
replacing it with the image-guided version described in \Cref{sec:basealg}.
\Cref{sec:tcmot} contains the motivation, i.e. why time coherence is beneficial,
and provide an example for undesirable effects that can appear without it.
The definition is provided in \Cref{sec:tcdef},
which we then adapt to our use case and implement in \Cref{sec:tcalg}.

\section{Initial Greedy Algorithm}
\label[section]{sec:failedalg}
\begin{leftbar}
    \noindent
    \textbf{Motivation:} 
    Since time coherence only exists as a connection from one set of streamlines to another, we started by creating an algorithm capable of generating such streamlines.
    We could then use this algorithm to generate sets of streamlines for time frames without time coherence as a negative example and to illustrate the effects of its absence.
    Initially, the algorithm was developed for the 2D case, the extension to 3D is shown in subsection 4.1.2.
\end{leftbar}
\noindent
The algorithm uses two operations, streamline traversal and seed filtering,
which are executed round-robin until there is no more room for new streamlines.
The result of the algorithm is a space-filling set of streamlines with even spacing in 3D fields.\\
We use 2 essential parameters to control how the images are generated.
The first one, $d_s$, is the neighbor search distance. 
It controls the distance from the current streamline where new streamlines start at,
allowing us to fill the whole space.
In order to add longer liens while guaranteeing even spacing,
we introduce a second parameter $d_c$, which is the neighbor cutoff distance.
As soon as a streamline comes within this distance to another streamline, it is terminated.\\
The typical range for $d_c$ is $0.5\,d_s \leq d_c \leq 0.75\,d_s$.
Making $d_c$ smaller than $0.5\,d_s$ allows streamlines to get very close,
introducing visual clutter and a crowded appearance.
At values $\geq0.75\,d_s$, a lot of streamlines are very short in forward or backward time due to the proximity of $d_s$ and $d_c$,
causing most of them to be removed immediately.

\newpage
\subsection{Two-Dimensional Implementation}
\begin{figure}[ht]
    \centering
    \begin{subfigure}[b]{0.3\textwidth}
        \centering
        \begin{tikzpicture}[domain=0:4, scale=.8]
    \draw[very thin,color=gray] (-0.1,-0.1) grid (3.9,3.9);

    \draw[->] (-0.2,0) -- (4.2,0) node[right] {$x$};
    \draw[->] (0,-0.2) -- (0,4.2) node[above] {$y$};

    \draw[dotted, thick]                       (1.5,0) -- (1.5,1) -- (1.73,1.87) -- (2.3,2.78) -- (3,3.4) -- (4,4);
    \draw plot[only marks, mark=x] coordinates{(1.5,0)               (1.73,1.87)    (2.3,2.78)    (3,3.4)};
    
    \draw[red] plot[only marks, mark=*] coordinates{      (1.5,1)};
\end{tikzpicture}

        \caption*{(a)}
    \end{subfigure}
    \begin{subfigure}[b]{0.3\textwidth}
        \centering
        \input{Sources/Diagrams/Implementation1_2}
        \caption*{(b)}
    \end{subfigure}
    \begin{subfigure}[b]{0.3\textwidth}
        \centering
        \includegraphics[scale=.1]{figures/OldAlgRotate.png}
        \caption*{(c)}
    \end{subfigure}
    \caption{
        (a): Forward and backward streamline integration through the field starting at the seed (red).
        The chosen sample points are drawn as a cross, and include the seed.
        (b): Equidistant seed candidates (green) obtained from the normals at the sample points are chosen for the next iteration.
        (c): Lines with equal distance generated by our algorithm in a field defined by $u(x,y)=(y, -x)$.}
    \label[figure]{fig:failedbasics}
\end{figure}
\vspace{-1cm}
\subsubsection{Streamline Traversal}
First, we choose a seed point from a list of candidates (initially an arbitrary point from the dataset) and remove it from the list.
Then we integrate forward and backward (\cref{fig:failedbasics} (a)) to obtain the other points on the streamline,
until a number of steps is reached, we cross the bounding box, or we get too close to another streamline.
If the total length of the streamline is too small, we remove it.\\
To obtain the seed candidates, we compute the normals of the field at these points, and add points a distance $d_s$ away from them to the list (\cref{fig:failedbasics} (b)).
\subsubsection{Seed Filtering}
The number of neighbor seed candidates is roughly 2$n$, where $n$ is the number of samples of the streamline we are integrating.
To quickly remove seeds that cannot produce a good streamline, we use two filtering conditions.
The first condition arises from the fact that roughly half the seeds generated by a streamline $L_1$ during the traversal process will lie close to - if not exactly on - the preceding streamline, $L_0$.
This happens because the seeds are placed a distance $d_s$ away, 
while the streamlines are mostly that same distance apart themselves.
The second condition is the distance to other seeds,
as starting a streamline from a seed that is too close to another will immediately end it.
Therefore, we filter two sets of seeds: The ones generated during the traversal process and those present in the entire field.
\begin{figure}[ht]
    \centering
    \begin{subfigure}[b]{0.4\textwidth}
        \centering
        \includegraphics[scale=.1]{figures/OldAlgOrbit.png}
        \caption*{(a)}
    \end{subfigure}
    \begin{subfigure}[b]{0.4\textwidth}
        \centering
        \includegraphics[scale=.3]{figures/OldAlgOrbitDistsZoom.png}
        \caption*{(b)}
    \end{subfigure}
    \caption{
        (a): The streamlines created in a double gyro dataset.
        (b): Top center view of (a). New streamlines created from the center streamline (red) are cut off due to reaching $d_c$ (magenta). New streamlines are only drawn at distance grater than $d_c$, producing the gap between the upper and lower three streamlines (black).
        The lower streamlines continue at a distance of $d_S$ (green).
    }
\end{figure}
\subsubsection{Line Cutoff and Optimization}
In order to quickly filter the $2n$-pairs for points on the parent streamline, we store the parent's points and remove any of the $2n$ seeds from the current step if they are too close.
When removing other seeds we come across, we employ a grid with a spacing of $d_c$ and use a dictionary to obtain lists of points inside the grid cell.
If we add a point during integration, we simply look up the coordinate and its 8 neighboring cells as a key to points in the area.
If a seed closer than $d_c$ is found in those cells, it is subsequently removed.
This allows for an easy way to end the integration when getting too close to another streamline as well:
We need only keep track of whether we came close to another streamline's points during integration
(provided the integration step size is less than $d_c$, which is usually the case).

\newpage
\subsection{Steady Field Streamline Placement in 3D}
\label[section]{sec:3D}
% The most important part to generate streamlines in 3D is obtaining the seed locations.
% In three dimensions, a vector has infinitely many normals, which all lie on a plane it is a normal of.
% Therefore, we define a number of points to evenly distribute around the streamline.
% The process of obtaining these points is as follows.
% \rule[2mm]{\textwidth}{0.4pt}
% \begin{minipage}{.6\textwidth}
%     Instead of the two trivial normals in 2D, we now construct a normal plane around the streamline trajectory at the streamline's points.
%     For this, we find two vectors which are linearly independent of each other and the trajectory, and then use NumPy's QR-decomposition to orthonormalize them.
%     The algorithm returns three orthonormalized column vectors, the first of which has a direction equal to the first provided input vector.
%     We can therefore feed it the streamline segment's direction and two basis vectors, and receive two orthonormal basis vectors $b_0, b_1$.
% \end{minipage}
% \begin{minipage}{.39\textwidth}
%     \centering
%     \setlength\pgfplotswidth{1.2\textwidth}
%     \begin{tikzpicture}
    \begin{axis}[ 
        ticks=none,
        axis lines = middle,
        axis line style={->},
        ymin=-.6, ymax=2.1,
        xmin=-.6, xmax=2.1,
        xlabel={$x$},
        ylabel={$z$},
        width=\pgfplotswidth,
        height=\pgfplotswidth
        ]
        \draw[rotate around={45:(.5,1)}] (axis cs:.5,1) ellipse (.6 and 1) [radius=1, fill=white];
        % Segment
        \draw[rotate around={45:(.5,1)}, ->, dotted, thick, red] (.5, 1) -- (1.5, 1) node[right] {$s$};
        % b0, b1
        \draw[rotate around={45:(.5,1)}, ->, blue!50] (.5, 1) -- (-.05, .85) node[right] {$b_0$};
        \draw[rotate around={45:(.5,1)}, ->, blue!50] (.5, 1) -- (.68, .09) node[right] {$b_1$};
      
        \draw[->] (0, 0) -- (-.5, -.5) node[right] {$y$};
    \end{axis}
\end{tikzpicture}

% \end{minipage}
% \rule{\textwidth}{0pt}
% \rule[3mm]{\textwidth}{0.4pt}
% \begin{minipage}{.6\textwidth}
%     Having found $b_0$ and $b_1$, we can use the roots of unity to find evenly spaced points on the unit circle.
%     With $i$ being the complex number, we obtain $k$ roots of unity via 
%     \vspace{-2mm}
%     \[n_j = e^{ji2\pi/k}, j = 0, 1, ..., k-1\]
%     \vspace{-2mm}
%     Since the magnitude of $n_j$ is always one, we do a simple basis transformation into the 3D frame of reference.
%     We obtain the $j$-th 3D-vector $v_j$ from the $j$-th root using our basis vectors $b_0$ and $b_1$:
%     \vspace{-2mm}
%     \[v_j = re(n_j)*b_0 + im(n_j)*b_1 \]
% \end{minipage}
% \begin{minipage}{.39\textwidth}
%     \centering
%     \setlength\pgfplotswidth{1.2\textwidth}
%     \begin{tikzpicture}
    \tikzset{
        pics/carc/.style args={#1:#2:#3}{
            code={
               \draw[pic actions] (#1:#3) arc(#1:#2:#3);
            }
        }
    }
    \begin{axis}[ 
        ticks=none,
        axis lines = middle,
        axis line style={->},
        ymin=-1.3, ymax=1.3,
        xmin=-1.3, xmax=1.3,
        xlabel={$1$},
        ylabel={$i$},
        width=\pgfplotswidth,
        height=\pgfplotswidth
        ]
        \draw (0,0) circle [radius=1, fill=white];
        \draw[very thick] (0,0) pic[red]{carc=0:72:2cm};
        \node[red] at (1,1) {$\frac{2\pi}{5}$};

        \draw[very thick, ->] (0,0) -- (1,0) node[midway, above right] {$n_0$};
        \draw[very thick, ->] (0,0) -- ( .309, .951) node[midway, right] {$n_1$};
        \draw[very thick, ->] (0,0) -- (-.809, .588) node[midway, above] {$n_2$};
        \draw[very thick, ->] (0,0) -- (-.809,-.588) node[midway, below] {$n_3$};
        \draw[very thick, ->] (0,0) -- ( .309,-.951) node[midway, right] {$n_4$};
    \end{axis}
\end{tikzpicture}

% \end{minipage}
% \rule[-1mm]{\textwidth}{0.4pt}
% This gives us $k$ uniformly placed vectors in the normal plane around the current streamline segment.
% The rest of the algorithm stays largely the same as in 2D, the grid used for seed filtering is extended into the 3rd dimension accordingly.
% See figure 4.5 on the next page for an example using a 3D vector field.
% \newpage
% \begin{figure}[ht]
%     \centering
%     \begin{subfigure}[b]{0.5\textwidth}
%         \centering
%         \includegraphics[scale=.1]{figures/OldAlg3D5.png}
%         \caption*{(a)}
%     \end{subfigure}
%     \begin{subfigure}[b]{0.4\textwidth}
%         \centering
%         \includegraphics[scale=.09]{figures/OldAlg3D400.png}
%         \caption*{(b)}
%     \end{subfigure}
%     \caption{
%         (a): A clear view of the center streamline with five neighbor streamlines evenly distributed at distance $d_S$.
%         (b): A filled cube containing 254 streamlines. Notice the resemblance to Figure 4.1 (c).
%     }
% \end{figure}
\begin{figure}[ht]
    \centering
    \begin{subfigure}{.33\textwidth}
        \centering
        \setlength\pgfplotswidth{\textwidth}
        \begin{tikzpicture}
    \begin{axis}[ 
        ticks=none,
        axis lines = middle,
        axis line style={->},
        ymin=-.6, ymax=2.1,
        xmin=-.6, xmax=2.1,
        xlabel={$x$},
        ylabel={$z$},
        width=\pgfplotswidth,
        height=\pgfplotswidth
        ]
        \draw[rotate around={45:(.5,1)}] (axis cs:.5,1) ellipse (.6 and 1) [radius=1, fill=white];
        % Segment
        \draw[rotate around={45:(.5,1)}, ->, dotted, thick, red] (.5, 1) -- (1.5, 1) node[right] {$s$};
        % b0, b1
        \draw[rotate around={45:(.5,1)}, ->, blue!50] (.5, 1) -- (-.05, .85) node[right] {$b_0$};
        \draw[rotate around={45:(.5,1)}, ->, blue!50] (.5, 1) -- (.68, .09) node[right] {$b_1$};
      
        \draw[->] (0, 0) -- (-.5, -.5) node[right] {$y$};
    \end{axis}
\end{tikzpicture}

        \caption*{(a)}
    \end{subfigure}
    \begin{subfigure}{.33\textwidth}
        \centering
        \includegraphics[scale=.05]{figures/OldAlg3D5.png}
        \caption*{(b)}
    \end{subfigure}
    \begin{subfigure}{.3\textwidth}
        \centering
        \includegraphics[scale=.05]{figures/OldAlg3D400.png}
        \caption*{(c)}
    \end{subfigure}
    \caption{
        (a): The normal plane of a streamline segment $s$ with two orthonormal basis vectors $b_0, b_1$.
        (b): A clear view of the center streamline with five neighbor streamlines evenly distributed at distance $d_S$.
        (c): A filled cube containing 254 streamlines. Notice the resemblance to \cref{fig:failedbasics} (c).
    }
    \label[figure]{fig:failed3d}
\end{figure}
The most important part to generate streamlines in 3D is obtaining the seed locations.
In three dimensions, a vector has infinitely many normals, which all lie on a plane it is a normal of.
Therefore, we define a number of points to evenly distribute around the streamline.
The process of obtaining these points is as follows.
Instead of the two trivial normals in 2D, we now construct a normal plane around the streamline trajectory at the streamline's points.
For this, we find two vectors which are linearly independent of each other and the trajectory,
and then orthonormalize them to receive two orthonormal basis vectors $b_0, b_1$ (see \cref{fig:failed3d} (a)).
Having found $b_0$ and $b_1$, we can use the roots of unity (fundamentals, \cref{rootsofunity}) to find evenly spaced points on the unit circle.
With $i$ being the complex number, we obtain $k$ roots of unity via 
\[n_j = e^{ji2\pi/k}, j = 0, 1, ..., k-1\]
Since the magnitude of $n_j$ is always one, we do a simple basis transformation into the 3D frame of reference.
We obtain the $j$-th 3D-vector $v_j$ from the $j$-th root using our basis vectors $b_0$ and $b_1$:
\[v_j = re(n_j)*b_0 + im(n_j)*b_1 \]
This gives us $k$ uniformly placed vectors in the normal plane around the current streamline segment.
The rest of the algorithm stays largely the same as in 2D, the grid used for seed filtering is extended into the 3rd dimension accordingly.
See \cref{fig:failed3d} (b, c) for an example using a 3D vector field.
\newpage
\begin{figure}[ht]
    \centering
    
\end{figure}
\subsubsection{Shortcomings}
\begin{figure}[ht]
    \begin{adjustwidth}{-2cm}{-2cm}
        \centering
        \begin{subfigure}[b]{.5\textwidth}
            \includegraphics[scale=.075]{figures/OAH2D1.png}
            \caption*{(a)}
        \end{subfigure}
        \begin{subfigure}[b]{.5\textwidth}
            \includegraphics[scale=.075]{figures/OAH2D2.png}
            \caption*{(b)}
        \end{subfigure}
    \end{adjustwidth}
    \caption{
        Two different line placements due to a seed choice difference of only $1e-4\%$ of image width.
    }
    \label[figure]{fig:failedshift}
\end{figure}
While this algorithm can quickly fill a space with decent streamlines,
the main problem is the inability to cope with small changes to streamlines.
This is mainly a result of the strong hierarchical nature:
Since every streamline after the first comes from its predecessor,
a change at the "root" can have drastic consequences for the succeeding streamlines (see \cref{fig:failedshift}).
In the context of time coherence, this makes the algorithm unsustainable, as an unsteady field is practically guaranteed to change at least slightly in the whole domain.
To make the streamlines time coherent, we need to be able to move streamlines around without affecting the global scope too much.
With this approach, changing the position of a streamline after the generation of its neighbors causes a re-evaluation of every streamline it is a predecessor of.
Due to this problem, we do not see an effective way to implement time coherence, and supersede this algorithm in favor of an image-guided one.
\newpage

\section{The Image-Guided Algorithm by Turk and Banks}
\label[section]{sec:basealg}
\begin{figure}[ht]
    \centering
    \begin{subfigure}{.49\textwidth}
        \centering
        \includegraphics*[scale=.09]{figures/TBWaves1.png}
        \caption*{(a)}
    \end{subfigure}
    \begin{subfigure}{.49\textwidth}
        \centering
        \includegraphics*[scale=.09]{figures/TBWaves2.png}
        \caption*{(b)}
    \end{subfigure}
    \caption{
        (a): Lines placed with seeds on a regular grid.
        (b): The same lines after optimization through seed shifting. Notice the more even grayscale image in the center and on the sides.
        Both images contain 130 lines, with a length of 10\% of the screen width each.}
    \label[figure]{fig:tbwaves}
\end{figure}

In this chapter, we introduce the image-guided streamline placement algorithm developed by Greg Turk and David Banks.
The algorithm forms a more suitable basis for time coherence due to how it reacts to changes to specific lines.
More specifically, it retains a good amount of control regarding the line placement and length,
while still allowing the lines to move and relax the spacing between them.

\subsection{Overview}
The following section will be about the three core components that make up the image-guided sections of the algorithm: the lines themselves, the low-pass filter, and the energy measure.
The filter is used to create a blurred image of the streamlines,
where the lines themselves are brighter, and empty spaces darker (\cref{fig:tbwaves}(a)).
An optimal line placement w.r.t. density (neither too crowded, nor too sparse) is reached when the low-pass image has roughly the same gray coloring everywhere.
We can intuitively define the energy measure as the squared differences of the blurred image from
a uniform grayscale background, also referred to as the \textit{target brightness}.
Simply knowing the energy however is not enough, we need to be able to manipulate the line placement effectively in order to minimize it.
This is achieved by using various randomized changes to the position (\cref{fig:tbwaves}(b)), length, or number of streamlines, and is described in greater detail in section 4.2.3.
Whenever a change leads to a decrease in energy, the change is accepted and, if not, reverted.

\subsection{Energy Measure}
The method used by Turk and Banks defines three important components to measure image quality as the sum of deviations of a low-pass image from a uniform greyscale target.
\begin{enumerate}
    \item The first component is a collection of (straight) line segments from each line,
    each of which can be converted to a line formula of the form $Ax + By - C = 0$.
    They refer to the image defined by these purely analytical lines as $I$, however this image is never actually created and exists purely implicitly.
    \begin{equation*}
        I(x,y) = \begin{cases}
            1, \kern4em & \text{pixel lies on line}\\
            0,          & \text{else}
    \end{cases}
    \end{equation*}
    
    \item The second component is the low-pass filter $L$.
    It uses a kernel to generate the filtered image of a line.
    Given a falloff distance $R$ and $r=\sqrt{x^2+y^2} / R$, the kernel is defined as:
    \begin{equation*}
        K(x,y) = \begin{cases}
            2r^3 - 3r^2 + 1, \kern4em & r < 1\\
            0,               & r >= 1
        \end{cases}
    \end{equation*}
    Every pixel within the bounding box of a segment + filter diameter has its brightness altered according to this kernel.
    Due to the implementation of the convolution and the scaling of the filter brightness, it is guaranteed that every pixel reaches a maximum brightness of 1.
    This is invariant w.r.t. the number or length of segments as long as only a single,
    straight line section is considered.
    Conversely, the only way to get a brightness grater than one is either via a curve in the line or two lines being next to each other.
    This is immensely useful, as it allows two lines to get close to each other from the ends, 
    but not from the sides (\cref{fig:tbwaves}(b)).
    
    \item In order to determine the energy of the image generated by the kernel application,
    the following expression (called the \textit{energy function}) is used:
    \begin{equation*}
        E(I) = \int_x\int_y\left[(L\ast I)(x,y)-t\right]^2\,dx\,dy
    \end{equation*}
    With $t$ referring to the \textit{target brightness}, in their source code the number one is used.
\end{enumerate}
It is pivotal for this energy measure to react to minimal changes for the algorithm to successfully converge.
Therefore, this algorithm does \textit{not} use a simple rasterization and blur technique, as the initial discretization/rasterization
of the line would remove too much information, even when using anti-aliasing or other more sophisticated rasterization techniques like Bresenham's line algorithm.
While we discuss the actual implementation in chapter 5, the takeaway here is that the line remains in its analytical form as segments between points.
Then, every point inside the bounding box around every line segment calculates its brightness
based on the line equation, not by simply blurring pixels on the line, leading to much higher sensitivity.

\subsection{Randomized Optimizations}
Turk and Banks define six actions:
\begin{itemize}
    \item \textbf{Insert, Delete:} Add or remove a line from the image.
    \item \textbf{Lengthen, Shorten:} In-/Decrease the length of a line on one or both ends.
    \item \textbf{Combine:} Join two lines head-to-tail.
    \item \textbf{Move:} Translate the seed of a line by a small distance.
\end{itemize}
These actions are selected randomly with random parameters, then applied to a random line.
The algorithm terminates after an energy range was reached, or accepted changes become rare enough to not introduce changes anymore.\\
If the change was deemed beneficial according to a decrease in energy, it is accepted, otherwise the changes are reverted.
This causes a "drift" of the lines toward a more uniform energy level.
Naturally, this depends heavily on the choice of $t$. If $t$ were to be chosen closer to 2,
the image would become very crowded to reach the increased target gray level.
\newpage

\begin{figure}[ht!]
    \centering
    \begin{subfigure}[b]{.49\textwidth}
        \centering
        \includegraphics[scale=.07]{figures/Necessary/TBWavesShort1.png}
        \caption*{(a)}
    \end{subfigure}
    \begin{subfigure}[b]{.49\textwidth}
        \centering
        \includegraphics[scale=.07]{figures/Necessary/TBWavesShort2.png}
        \caption*{(b)}
    \end{subfigure}
    \caption{
        (a), (b): A vector field undergoing a change over time, increasing the amplitude at its center, visualized using
        a hedgehog plot.
        It is described by $u(x,y,t) = (1, \sin(5t\pi x) \text{ if } 0.4 \leq x \leq 0.6 \text{ else } 0)^T$ for $t=1$ and $t=3$ in (a) and (b) respectively.
    }
    \label[figure]{fig:necessary1}
\end{figure}

\section{Motivation for Time Coherence}
\label[section]{sec:tcmot}
When visualizing a vector field using streamlines, it is desirable to capture the features of the field as accurately as possible.
For image-based solutions, this means that we are interested in a line placement that poses little to no visual distractions from the general field flow.
Common distractions include artifacts like shapes arising from the position of lines, but not being created by the actual field.
Other forms of visual clutter include strong variations in density, or empty spaces, which make it very hard to judge a field's behavior in these regions.
Usually, there is a trade-off between the accuracy of capturing a field and visual clarity,
with image-guided approaches favoring the latter.

A new type of visual artifact is introduced when we add a time axis to our data.
Where before, this approach could solely focus on optimizing a single image w.r.t. visual criteria, we now have to take into account the movement
of lines from one time step to another.

We can illustrate this behavior using the field in \Cref{fig:necessary1},
taking a closer look at $u(x,y,t)$ at two time steps $t=1$ and $t=3$.
The notable change of the field's trajectory from one time step to the next
is a strong ridge forming in the center column.

\newpage

\begin{figure}[ht!]
    \centering
    \begin{subfigure}[b]{.49\textwidth}
        \centering
        \includegraphics[scale=.07]{figures/Necessary/TBWavesLong1.png}
        \caption*{(a)}
    \end{subfigure}
    \begin{subfigure}[b]{.49\textwidth}
        \centering
        \includegraphics[scale=.07]{figures/Necessary/TBWavesLong2.png}
        \caption*{(b)}
    \end{subfigure}
    \caption{
        A streamline (black) seeded directly at the center for both cases.
        The colored arrows represent the perceived direction of motion.
        (a) The streamline mostly drifting downward from time $t=1$ to $t=3$ (red).
        (b) The same initial streamline at $t=1$, but keeping same height at time $t=3$ (green) because the algorithm shifted the seed.
    }
    \label[figure]{fig:necessary2}
\end{figure}
We now show how this behavior can lead to a visual artifact when using streamlines as the means of visualization.
In \Cref{fig:necessary2} (a), we see how the increase in ridge height leads to movement of the whole streamline,
which is visually irritating as it distracts from the actual feature undergoing the change.
In fact, it suggests that instead of the center moving upward, the outer regions move downward,
giving room for misconception of the field's behavior.

What we are looking for is a line placement strategy that allows large portions of the image to 
stay the same between two time steps,
i.e. that achieves the largest possible ``streamline overlap'' between the two images, as shown
in the transition in \Cref{fig:necessary2} (b).
Here, large portions remain stationary, with only segments close to the center rising upwards,
reflecting the localized change much better.

An important factor for the generation of such images is the seed choice.
We have chosen the same centered seed in both cases in order to compare the
placement with and without our algorithm optimizing the seed position to respect
what we define in the next chapter as \textit{time coherence}.

\newpage

\begin{figure}[ht!]
    \centering
    \begin{subfigure}{.3\textwidth}
        \centering
        \includegraphics[scale=.075]{figures/FilterRadius/Distant.png}
        \caption*{(a)}
    \end{subfigure}
    \begin{subfigure}{.3\textwidth}
        \centering
        \includegraphics[scale=.075]{figures/FilterRadius/Close.png}
        \caption*{(b)}
    \end{subfigure}
    \begin{subfigure}{.375\textwidth}
        \centering
        \resizebox{\textwidth}{!}{
            % Data for the first, distant pair of horizontal lines. Extracted along the y axis via target.data[20, :].round(3)

% [0.   0.   0.   0.   0.   0.   0.   0.   0.   0.   0.   0.   0.   0.
%  0.   0.   0.   0.   0.   0.   0.   0.   0.   0.   0.   0.   0.   0.
%  0.   0.   0.   0.   0.   0.   0.   0.   0.   0.   0.   0.04 0.11 0.23
%  0.38 0.55 0.71 0.86 0.97 1.03 1.03 0.97 0.86 0.71 0.55 0.38 0.23 0.11
%  0.04 0.   0.   0.   0.   0.   0.   0.04 0.11 0.23 0.38 0.55 0.71 0.86
%  0.97 1.03 1.03 0.97 0.86 0.71 0.55 0.38 0.23 0.11 0.04 0.   0.   0.
%  0.   0.   0.   0.   0.   0.   0.   0.   0.   0.   0.   0.   0.   0.
%  0.   0.   0.   0.   0.   0.   0.   0.   0.   0.   0.   0.   0.   0.
%  0.   0.   0.   0.   0.   0.   0.   0.  ]


\begin{tikzpicture}
    \begin{axis} [every axis plot post/.append style={ultra thick, mark=none}, xlabel=]
        \addplot +[smooth] [color=blue] coordinates {
            (0, 0.)
            (1, 0.)
            (2, 0.)
            (3, 0.)
            (4, 0.)
            (5, 0.)
            (6, 0.)
            (7, 0.)
            (8, 0.)
            (9, 0.)
            (10, 0.)
            (11, 0.)
            (12, 0.)
            (13, 0.)
            (14, 0.)
            (15, 0.)
            (16, 0.)
            (17, 0.)
            (18, 0.)
            (19, 0.)
            (20, 0.)
            (21, 0.)
            (22, 0.)
            (23, 0.)
            (24, 0.)
            (25, 0.)
            (26, 0.)
            (27, 0.)
            (28, 0.)
            (29, 0.)
            (30, 0.)
            (31, 0.)
            (32, 0.)
            (33, 0.)
            (34, 0.)
            (35, 0.)
            (36, 0.)
            (37, 0.)
            (38, 0.)
            (39, 0.04)
            (40, 0.11)
            (41, 0.23)
            (42, 0.38)
            (43, 0.55)
            (44, 0.71)
            (45, 0.86)
            (46, 0.97)
            (47, 1.03)
            (48, 1.03)
            (49, 0.97)
            (50, 0.86)
            (51, 0.71)
            (52, 0.55)
            (53, 0.38)
            (54, 0.23)
            (55, 0.11)
            (56, 0.04)
            (57, 0.)
            (58, 0.)
            (59, 0.)
            (60, 0.)
            (61, 0.)
            (62, 0.)
            (63, 0.04)
            (64, 0.11)
            (65, 0.23)
            (66, 0.38)
            (67, 0.55)
            (68, 0.71)
            (69, 0.86)
            (70, 0.97)
            (71, 1.03)
            (72, 1.03)
            (73, 0.97)
            (74, 0.86)
            (75, 0.71)
            (76, 0.55)
            (77, 0.38)
            (78, 0.23)
            (79, 0.11)
            (80, 0.04)
            (81, 0.)
            (82, 0.)
            (83, 0.)
            (84, 0.)
            (85, 0.)
            (86, 0.)
            (87, 0.)
            (88, 0.)
            (89, 0.)
            (90, 0.)
            (91, 0.)
            (92, 0.)
            (93, 0.)
            (94, 0.)
            (95, 0.)
            (96, 0.)
            (97, 0.)
            (98, 0.)
            (99, 0.)
            (100, 0.)
            (101, 0.)
            (102, 0.)
            (103, 0.)
            (104, 0.)
            (105, 0.)
            (106, 0.)
            (107, 0.)
            (108, 0.)
            (109, 0.)
            (110, 0.)
            (111, 0.)
            (112, 0.)
            (113, 0.)
            (114, 0.)
            (115, 0.)
            (116, 0.)
            (117, 0.)
            (118, 0.)
            (119, 0.)
        };
        \addplot +[smooth] [color=red] coordinates {
            (0, 0.)
            (1, 0.)
            (2, 0.)
            (3, 0.)
            (4, 0.)
            (5, 0.)
            (6, 0.)
            (7, 0.)
            (8, 0.)
            (9, 0.)
            (10, 0.)
            (11, 0.)
            (12, 0.)
            (13, 0.)
            (14, 0.)
            (15, 0.)
            (16, 0.)
            (17, 0.)
            (18, 0.)
            (19, 0.)
            (20, 0.)
            (21, 0.)
            (22, 0.)
            (23, 0.)
            (24, 0.)
            (25, 0.)
            (26, 0.)
            (27, 0.)
            (28, 0.)
            (29, 0.)
            (30, 0.)
            (31, 0.)
            (32, 0.)
            (33, 0.)
            (34, 0.)
            (35, 0.)
            (36, 0.)
            (37, 0.)
            (38, 0.)
            (39, 0.)
            (40, 0.)
            (41, 0.)
            (42, 0.)
            (43, 0.)
            (44, 0.)
            (45, 0.)
            (46, 0.)
            (47, 0.003)
            (48, 0.037)
            (49, 0.115)
            (50, 0.234)
            (51, 0.384)
            (52, 0.549)
            (53, 0.716)
            (54, 0.894)
            (55, 1.081)
            (56, 1.26)
            (57, 1.41)
            (58, 1.516)
            (59, 1.571)
            (60, 1.571)
            (61, 1.516)
            (62, 1.41)
            (63, 1.26)
            (64, 1.081)
            (65, 0.894)
            (66, 0.716)
            (67, 0.549)
            (68, 0.384)
            (69, 0.234)
            (70, 0.115)
            (71, 0.037)
            (72, 0.003)
            (73, 0.)
            (74, 0.)
            (75, 0.)
            (76, 0.)
            (77, 0.)
            (78, 0.)
            (79, 0.)
            (80, 0.)
            (81, 0.)
            (82, 0.)
            (83, 0.)
            (84, 0.)
            (85, 0.)
            (86, 0.)
            (87, 0.)
            (88, 0.)
            (89, 0.)
            (90, 0.)
            (91, 0.)
            (92, 0.)
            (93, 0.)
            (94, 0.)
            (95, 0.)
            (96, 0.)
            (97, 0.)
            (98, 0.)
            (99, 0.)
            (100, 0.)
            (101, 0.)
            (102, 0.)
            (103, 0.)
            (104, 0.)
            (105, 0.)
            (106, 0.)
            (107, 0.)
            (108, 0.)
            (109, 0.)
            (110, 0.)
            (111, 0.)
            (112, 0.)
            (113, 0.)
            (114, 0.)
            (115, 0.)
            (116, 0.)
            (117, 0.)
            (118, 0.)
            (119, 0.)
        };
        \addlegendentry{(a)'s energy}
        \addlegendentry{(b)'s energy}
    \end{axis}
\end{tikzpicture}
        }
        \caption*{(c)}
    \end{subfigure}
    \caption{
        How the proximity of streamlines changes the brightness of their respective footprints in a 120x120\,px image.
        (a) Two streamlines approx. three times the blur radius apart.
        (b) Two streamlines only 3/4 the blur radius apart.
        (c) Energy of the more distant streamlines (a, blue) and the closer ones (b, red) is shown in the $y$-axis.
        The $x$-axis shows the height of the pixels taken along the red and blue
        strips in (a) and (b) respectively, counted from the top.
    }
    \label[figure]{fig:closeenergy}
\end{figure}
\section{Time Coherence - Definition}
\label[section]{sec:tcdef}
Based on the desired image overlap and concepts from computing the energy measure, we can infer a measure for time coherence.
We base this measure on the spatial energy function $E$ from the Turk and Banks algorithm.
\begin{leftbar}
    \textbf{Notation:} To avoid confusion, we from now on write the spatial components (former $E$ and $L$)
    as $E_s, L_s$ to better separate them from their temporal counterparts $E_t$ and $L_t$.
    $t$ has two further uses:
    If we talk about time steps, $t$ refers to the time, e.g., when describing vector fields $u(x,y,t)$.
    In the context of the Turk and Banks algorithm, $t$ is the target brightness.
\end{leftbar}

Instead of the comparison between a low-pass image of streamlines and a constant target brightness used in $E_s$,
the temporal energy $E_t$ now depends on two sets of streamlines ($I_0, I_1$) and uses a different low-pass filter ($L_t$):
\[E_t(I_0, I_1) = \int_x\int_y\left[(L_t\ast I_0)(x,y)-(L_t\ast I_1)(x,y)\right]^2\,\text{d}x\,\text{d}y\]
This measure tells us the sum of squared difference between the energy of two different sets of streamlines.
Allowing for a new kernel gives us more freedom to change \textit{how} we measure the temporal energy,
as we do not want it to behave the same way as the spatial energy does.
For example, the measure $E_s$ of two neighboring streamlines is the strongest in
the center of them, not at the actual streamline positions, which can be seen in \Cref{fig:closeenergy}.
This means that by using $E_s$ instead of $E_t$,
we would change the number or position of streamlines by drawing them into the center of two former streamlines,
thereby intently worsening time coherence.
Conversely, this would also prohibit streamline creation at darker spots, giving rise to holes in our streamline image.

The optimum for time coherence would of course be two identical sets of streamlines, which effectively minimize $E_t$ to zero.

% Combining it with the spatial measure $E$ via linear interpolation gives us a good amount of control how strong we want the coherence to be.

% Time Coherence refers to how a vector field behaves through different time steps.
% Intuitively, we consider areas within the field to be of high temporal coherence if the lines drawn on them are relatively stationary.
% Vice versa, we can say that an area of high fluctuation will be of low temporal coherence.
% A more formal definition employed in our algorithm is as follows:
% Given a field $F$ and a starting point $S_0$ (called the "seed"), we can integrate over the field.
% This yields a set of points $S^0$ which define a streamline containing every reached point, written as $S^0 = \int(S_0, F)$.
% We can therefore assign a streamline to every point in our field (and vice versa).
% Given $S_0$ and an unsteady field $F(t)$, compute for each time step $t_1...t_n$ the streamline $S^{0,t_i} = \int(S_0, F(t_i))$.
% In order to convert these sets of lines to a scalar, we use the Hausdorff Distance $dist(S^i,S^j)$,
% giving us the greatest minimal distance between any pair of two sets.
% We can therefore create a map $coh(S_i, F(t)): max(dist(\int(S_i, F(t_k)), \int(S_i, F(t_l))))$,
% sending each point in an unsteady vector field to a scalar, and thereby determining its temporal coherence.

\section{Adding Time Coherence}
\label[section]{sec:tcalg}
This section will be about how we translated the definition from \Cref{sec:tcdef} into a functional component for our algorithm.
We use the aforementioned energy measure to induce a process we refer to as \textit{coaxing},
as we are applying continued pressure to streamlines, moving them in the direction of their former footprints.

A second addition we call \textit{shattering} will be introduced as well,
where streamlines are split into smaller segments (\textit{fragments}) 
to be used as the starting layout for the next time step.
This can increase coherence as well by seeding streamlines at the same place,
and works especially well in combination with coaxing.

\subsection{Coaxing}
Since most of the algorithm's optimization is centered around the comparison
with an energy level before and after an action was taken,
modifying the energy function provides good leverage regarding how streamlines are placed. 
In order to make the algorithm favor previous streamline positions,
we therefore rewrite the previous energy function $E$ as the linear
interpolation between $E_s$ and $E_t$. This gives us good control over how much time coherence we apply, as choosing too much will cause a degradation in image quality.
Given the previous frame's low-pass image generated using the time kernel $L_t\ast I_0$ and the current image as $L_t\ast I_1$, we use:
\begin{equation*}
    \begin{split}
        E(I_0, I_1) &= \alpha E_s(I)+(1-\alpha)E_t(I_0, I_1)\\
        E_s(I_1)      &= \int_x\int_y\left[(L_s\ast I_1)(x,y)-t\right]^2\,\text{d}x\,\text{d}y\\
        E_t(I_0, I_1)  &= \int_x\int_y\left[(L_t\ast I_0)(x,y)-(L_t\ast I_1)(x,y)\right]^2\,\text{d}x\,\text{d}y
    \end{split}
\end{equation*}
We have found values for $\alpha$ in the range $[0.4, 0.8]$ to be effective.
Choosing a higher value causes very few streamlines to be drawn,
and only yields sporadic segments due to the inhibitory effect on the lengthen and
join operations when leaving the previous streamline's footprint.
This gets exacerbated by the gaps between the fragments being cemented in the new $L_t\ast I$,
not allowing them to reconnect in subsequent frames.

\begin{figure}[t]
    \centering
    \begin{subfigure}{.24\textwidth}
        \centering
        \includegraphics[scale=.065]{figures/Coaxing/Glyphs.png}
        \caption{}
    \end{subfigure}
    \begin{subfigure}{.24\textwidth}
        \centering
        \includegraphics[scale=.065]{figures/Coaxing/SingleLine0.png}
        \caption{}
    \end{subfigure}
    \begin{subfigure}{.24\textwidth}
        \centering
        \includegraphics[scale=.065]{figures/Coaxing/SingleLine1.png}
        \caption{}
    \end{subfigure}
    \begin{subfigure}{.24\textwidth}
        \centering
        \includegraphics[scale=.065]{figures/Coaxing/SingleLine2.png}
        \caption{}
    \end{subfigure}
    \caption{
        Streamlines may diverge from the same origin when using $\alpha=0$. The field in this figure is $steady$.
        (a) We use the double gyre to show divergent behavior in (d), visualized with an arrow plot.
        (b) Initial seed and starting streamline length.
        (c) Random move and lengthen steps reach a local minimum energy by increasing the streamline length, converging to one side at random.
        (d) A different result may occur with the same starting conditions.
    }
    \label[figure]{fig:energydevelopment}
\end{figure}
Since we compare many images and footprints from this section onward, it is useful to include a distinction using different color channels.
For the rest of this thesis, we use a consistent coloring to show streamline movement between time frames.
Footprints from the current frame's streamlines are drawn in green, those from the last step in red.
The higher the intensity of a pixel, the stronger the energy in that region.
High time coherence therefore leads to most of the image being yellow, with few red or green areas.
We only draw the footprints obtained using the filter $L_t$, as those from $L_s$ can be easily inferred while
$L_t$ provides more visual clarity due to the reduced blur radius.
\begin{leftbar}
    \textbf{Note}: The algorithm may perform a combination of move and lengthen operations at once.
    Even with a constant field, it is possible for streamlines to move or change their length slightly due to this inherent randomness.
    % Nonetheless, this section will give an accurate overview of how changing $\alpha$ impacts image quality.
\end{leftbar}

We now take a closer look at the energy development for different streamline positions
seen in \Cref{fig:energydevelopment} (a--d).
We start with a simple, steady field shown in (a), and a constant starting position for
every execution at the center (b).
After 100 optimization steps, the streamline has grown to the maximum length possible (c),
thereby reaching a minimum in spatial energy.
(d) shows the two likely outcomes of how the starting streamline develops under the specified starting conditions.
Due to the randomness of the algorithm, there is a $\approx50\,\%$ chance of ending up on either side of the center ridge (d).
\newpage

\begin{figure}[ht]
    \begin{subfigure}{\textwidth}
        \begin{subfigure}{.33\textwidth}
            \centering
            \includegraphics[scale=.06]{figures/Coaxing/SingleLine2.png}
            \caption*{\textbf{(a)}}
        \end{subfigure}
        \begin{subfigure}{.65\textwidth}
            \centering
            \begin{tikzpicture}
    \begin{axis} [every axis plot post/.append style={ultra thick, mark=none}, xlabel=]
        % Separate        
        \addplot +[smooth] [color=blue] coordinates {
            (0, 14007.826)
            (1, 13989.577)
            (2, 13856.987)
            (3, 13741.918)
            (4, 13681.778)
            (5, 13540.625)
            (6, 13445.154)
            (7, 13229.154)
            (8, 13151.793)
            (9, 13028.149)
            (10, 12979.961)
            (11, 12974.256)
            (12, 12974.256)
            (13, 12974.256)
            (14, 12974.256)
            (15, 12974.256)
            (16, 12974.256)
            (17, 12974.256)
            (18, 12974.256)
            (19, 12974.256)
            (20, 12974.256)
            (21, 12974.256)
            (22, 12974.256)
            (23, 12974.256)
            (24, 12968.2)
            (25, 12968.2)
            (26, 12965.026)
            (27, 12965.026)
            (28, 12965.026)
            (29, 12965.026)
            (30, 12965.026)
            (31, 12965.026)
            (32, 12965.026)
            (33, 12965.026)
            (34, 12965.026)
            (35, 12965.026)
            (36, 12965.026)
            (37, 12965.026)
            (38, 12965.026)
            (39, 12965.026)
            (40, 12965.026)
            (41, 12965.026)
            (42, 12965.026)
            (43, 12965.026)
            (44, 12959.116)
            (45, 12950.877)
            (46, 12950.877)
            (47, 12912.311)
            (48, 12874.413)
            (49, 12874.413)
            (50, 12874.413)
            (51, 12807.514)
            (52, 12807.514)
            (53, 12807.514)
            (54, 12807.514)
            (55, 12807.514)
            (56, 12807.514)
            (57, 12807.514)
            (58, 12807.514)
            (59, 12807.514)
            (60, 12801.792)
            (61, 12801.792)
            (62, 12801.792)
            (63, 12801.792)
            (64, 12801.289)
            (65, 12801.289)
            (66, 12801.289)
            (67, 12801.289)
            (68, 12801.289)
            (69, 12801.289)
            (70, 12801.289)
            (71, 12801.289)
            (72, 12801.289)
            (73, 12801.289)
            (74, 12801.289)
            (75, 12801.289)
            (76, 12801.289)
            (77, 12801.289)
            (78, 12801.289)
            (79, 12801.289)
            (80, 12801.289)
            (81, 12801.289)
            (82, 12801.289)
            (83, 12801.289)
            (84, 12801.289)
            (85, 12801.289)
            (86, 12801.289)
            (87, 12801.289)
            (88, 12801.289)
            (89, 12801.289)
            (90, 12801.289)
            (91, 12801.289)
            (92, 12801.289)
            (93, 12801.289)
            (94, 12801.289)
            (95, 12801.289)
            (96, 12801.289)
            (97, 12801.289)
            (98, 12801.289)
            (99, 12801.289)
        };
        % Overlapping
        \addplot +[smooth] [color=magenta] coordinates {
            (0, 14007.826) 
            (1, 13869.016) 
            (2, 13622.305) 
            (3, 13479.086) 
            (4, 13146.85)  
            (5, 13052.909) 
            (6, 13024.708) 
            (7, 12989.768) 
            (8, 12989.768) 
            (9, 12989.768) 
            (10, 12989.768)
            (11, 12951.466)
            (12, 12951.466)
            (13, 12951.466)
            (14, 12951.466)
            (15, 12951.466)
            (16, 12940.784)
            (17, 12940.784)
            (18, 12923.523)
            (19, 12923.523)
            (20, 12923.523)
            (21, 12923.523)
            (22, 12923.523)
            (23, 12923.523)
            (24, 12921.698)
            (25, 12921.698)
            (26, 12921.698)
            (27, 12921.698)
            (28, 12921.698)
            (29, 12921.698)
            (30, 12860.99) 
            (31, 12860.99) 
            (32, 12860.99)
            (33, 12827.076)
            (34, 12827.076)
            (35, 12827.076)
            (36, 12827.076)
            (37, 12827.076)
            (38, 12827.076)
            (39, 12827.076)
            (40, 12827.076)
            (41, 12827.076)
            (42, 12827.076)
            (43, 12827.076)
            (44, 12827.076)
            (45, 12827.076)
            (46, 12798.14)
            (47, 12798.14)
            (48, 12798.14)
            (49, 12798.14)
            (50, 12798.14)
            (51, 12798.14)
            (52, 12798.14)
            (53, 12798.14)
            (54, 12798.14)
            (55, 12798.14)
            (56, 12798.14)
            (57, 12798.14)
            (58, 12798.14)
            (59, 12798.14)
            (60, 12798.14)
            (61, 12798.14)
            (62, 12798.14)
            (63, 12798.14)
            (64, 12798.14)
            (65, 12798.14)
            (66, 12798.14)
            (67, 12798.14)
            (68, 12798.14)
            (69, 12798.14)
            (70, 12798.14)
            (71, 12798.14)
            (72, 12798.14)
            (73, 12798.14)
            (74, 12798.14)
            (75, 12798.14)
            (76, 12798.14)
            (77, 12798.14)
            (78, 12798.14)
            (79, 12798.14)
            (80, 12798.14)
            (81, 12798.14)
            (82, 12798.14)
            (83, 12798.14)
            (84, 12798.14)
            (85, 12798.14)
            (86, 12798.14)
            (87, 12798.14)
            (88, 12798.14)
            (89, 12798.14)
            (90, 12798.14)
            (91, 12798.14)
            (92, 12798.14)
            (93, 12798.14)
            (94, 12798.14)
            (95, 12798.14)
            (96, 12798.14)
            (97, 12798.14)
            (98, 12798.14)
            (99, 12798.14)
        };
        \addlegendentry{(d)'s spatial energy}
        \addlegendentry{(e)'s spatial energy}
    \end{axis}
\end{tikzpicture}

            \caption*{\textbf{(c)}}
        \end{subfigure}
    \end{subfigure}
    \begin{subfigure}{\textwidth}
        \begin{subfigure}{.33\textwidth}
            \centering
            \includegraphics[scale=.06]{figures/Coaxing/SingleLineC1.png}
            \caption*{\textbf{(b)}}
        \end{subfigure}
        \begin{subfigure}{.65\textwidth}
            \centering
            \begin{tikzpicture}
    \begin{axis} [
        every axis plot post/.append style={ultra thick, mark=none}, 
        xlabel=,
        legend style={at={(.98,.8)},anchor=north east}
    ]
        \addplot +[smooth] [color=blue] coordinates {
            (0, 600.738)
            (1, 606.401)
            (2, 647.542)
            (3, 683.247)
            (4, 700.14)
            (5, 747.113)
            (6, 776.618)
            (7, 846.143)
            (8, 870.048)
            (9, 908.393)
            (10, 928.066)
            (11, 929.862)
            (12, 929.862)
            (13, 929.862)
            (14, 929.862)
            (15, 929.862)
            (16, 929.862)
            (17, 929.862)
            (18, 929.862)
            (19, 929.862)
            (20, 929.862)
            (21, 929.862)
            (22, 929.862)
            (23, 929.862)
            (24, 931.591)
            (25, 931.591)
            (26, 932.786)
            (27, 932.786)
            (28, 932.786)
            (29, 932.786)
            (30, 932.786)
            (31, 932.786)
            (32, 932.786)
            (33, 932.786)
            (34, 932.786)
            (35, 932.786)
            (36, 932.786)
            (37, 932.786)
            (38, 932.786)
            (39, 932.786)
            (40, 932.786)
            (41, 932.786)
            (42, 932.786)
            (43, 932.786)
            (44, 934.333)
            (45, 936.797)
            (46, 936.797)
            (47, 949.954)
            (48, 963.133)
            (49, 963.133)
            (50, 963.133)
            (51, 986.039)
            (52, 986.039)
            (53, 986.039)
            (54, 986.039)
            (55, 986.039)
            (56, 986.039)
            (57, 986.039)
            (58, 986.039)
            (59, 986.039)
            (60, 988.596)
            (61, 988.596)
            (62, 988.596)
            (63, 988.596)
            (64, 989.078)
            (65, 989.078)
            (66, 989.078)
            (67, 989.078)
            (68, 989.078)
            (69, 989.078)
            (70, 989.078)
            (71, 989.078)
            (72, 989.078)
            (73, 989.078)
            (74, 989.078)
            (75, 989.078)
            (76, 989.078)
            (77, 989.078)
            (78, 989.078)
            (79, 989.078)
            (80, 989.078)
            (81, 989.078)
            (82, 989.078)
            (83, 989.078)
            (84, 989.078)
            (85, 989.078)
            (86, 989.078)
            (87, 989.078)
            (88, 989.078)
            (89, 989.078)
            (90, 989.078)
            (91, 989.078)
            (92, 989.078)
            (93, 989.078)
            (94, 989.078)
            (95, 989.078)
            (96, 989.078)
            (97, 989.078)
            (98, 989.078)
            (99, 989.078)    
        };
        \addplot +[smooth] [color=magenta] coordinates {
            (0, 599.266) 
            (1, 642.339) 
            (2, 716.691) 
            (3, 760.73)  
            (4, 851.884) 
            (5, 878.242) 
            (6, 886.961) 
            (7, 902.354) 
            (8, 902.354) 
            (9, 902.354) 
            (10, 902.354)
            (11, 883.831)
            (12, 883.831)
            (13, 883.831)
            (14, 883.831)
            (15, 883.831)
            (16, 858.702)
            (17, 858.702)
            (18, 777.924)
            (19, 777.924)
            (20, 777.924)
            (21, 777.924)
            (22, 777.924)
            (23, 777.924)
            (24, 778.653)
            (25, 778.653)
            (26, 778.653)
            (27, 778.653)
            (28, 778.653)
            (29, 778.653)
            (30, 486.235)
            (31, 486.235)
            (32, 486.235)
            (33, 163.149)
            (34, 163.149)
            (35, 163.149)
            (36, 163.149)
            (37, 163.149)
            (38, 163.149)
            (39, 163.149)
            (40, 163.149)
            (41, 163.149)
            (42, 163.149)
            (43, 163.149)
            (44, 163.149)
            (45, 163.149)
            (46, 0.216)
            (47, 0.216)
            (48, 0.216)
            (49, 0.216)
            (50, 0.216)
            (51, 0.216)
            (52, 0.216)
            (53, 0.216)
            (54, 0.216)
            (55, 0.216)
            (56, 0.216)
            (57, 0.216)
            (58, 0.216)
            (59, 0.216)
            (60, 0.216)
            (61, 0.216)
            (62, 0.216)
            (63, 0.216)
            (64, 0.216)
            (65, 0.216)
            (66, 0.216)
            (67, 0.216)
            (68, 0.216)
            (69, 0.216)
            (70, 0.216)
            (71, 0.216)
            (72, 0.216)
            (73, 0.216)
            (74, 0.216)
            (75, 0.216)
            (76, 0.216)
            (77, 0.216)
            (78, 0.216)
            (79, 0.216)
            (80, 0.216)
            (81, 0.216)
            (82, 0.216)
            (83, 0.216)
            (84, 0.216)
            (85, 0.216)
            (86, 0.216)
            (87, 0.216)
            (88, 0.216)
            (89, 0.216)
            (90, 0.216)
            (91, 0.216)
            (92, 0.216)
            (93, 0.216)
            (94, 0.216)
            (95, 0.216)
            (96, 0.216)
            (97, 0.216)
            (98, 0.216)
            (99, 0.216)   
        };
        \addlegendentry{(d)'s temp. energy}
        \addlegendentry{(e)'s temp. energy}
    \end{axis}
\end{tikzpicture}

            \caption*{\textbf{(d)}}
        \end{subfigure}
    \end{subfigure}
    \caption{
        (a) Two divergent streamlines resulting from the same initial streamlet.
        (b) The line overlap in favor of temporal energy.
        (c) Spatial energy of (a) and (b) vs optimization steps.
        (d) Temporal energy of (a) and (b) vs optimization steps.
    }
\label[figure]{fig:energydevelopment2}
\end{figure}
In \Cref{fig:energydevelopment2} (c), we see a plot of the spatial energy vs the current optimization step.
The maximum spatial energy equals the image resolution at 120x120=14400.
Initially, we see a decline in spatial energy for both curves, caused by the comparatively fast lengthening process.
On average, the streamlines grow about 3\,\% image size per step on both sides.
A plateau phase is reached after $\approx$ 10 steps,
as the streamlines approach the domain boundaries and cannot lengthen further.
Due to random movements, the streamline ends drift along the upper and lower domain borders,
slowly lengthening due to the field's curvature increasing the further outward they move.
A final stronger decline happens at 30 and 50 steps for (a) and (b) respectively, as the curvature is strong enough to allow larger regions to be filled by lengthening again.
The delay between (a) and (b) is caused by the random nature of the algorithm, and is equal if both runs use the same randomization setup.
Our streamlines reach the minimum possible energy of about 12800, with a reduction of $\approx 1600$.
Since the streamline length lies at roughly 140\,px and the filter is about 14\,px wide (note: $L_t$ as shown is half the size of $L_s$),
these spatial energy measures aren't surprising. The temporal energy is shown on plot (d), where we can see a stark difference in the development of (a) and (b).
Many features regarding local rate of change are recognizable in both plots,
e.g. the plateau of (b) centered around the 40 step mark.
The final temporal difference lies at 1000 due to the reduced filter radius.
\newpage

\begin{figure}[ht]
    \centering
    % 5 x No coherence
    \begin{subfigure}{\textwidth}
        \centering
        \begin{subfigure}{.19\textwidth}
            \centering
            \includegraphics[scale=.05]{figures/AlphaStudy/GyroL05.png}
        \end{subfigure}
        \begin{subfigure}{.19\textwidth}
            \centering
            \includegraphics[scale=.05]{figures/AlphaStudy/GyroA03L05.png}
        \end{subfigure}
        \begin{subfigure}{.19\textwidth}
            \centering
            \includegraphics[scale=.05]{figures/AlphaStudy/GyroA13L05.png}
        \end{subfigure}
        \begin{subfigure}{.19\textwidth}
            \centering
            \includegraphics[scale=.05]{figures/AlphaStudy/GyroA23L05.png}
        \end{subfigure}
        \begin{subfigure}{.19\textwidth}
            \centering
            \includegraphics[scale=.05]{figures/AlphaStudy/GyroA33L05.png}
        \end{subfigure}
    \end{subfigure}
    \begin{subfigure}{\textwidth}
        \centering
        \begin{subfigure}{.19\textwidth}
            \centering
            \includegraphics[scale=.05]{figures/AlphaStudy/GyroL05_Lines.png}
            \caption{}
        \end{subfigure}
        \begin{subfigure}{.19\textwidth}
            \centering
            \includegraphics[scale=.05]{figures/AlphaStudy/GyroA03L05_Lines.png}
            \caption{}
        \end{subfigure}
        \begin{subfigure}{.19\textwidth}
            \centering
            \includegraphics[scale=.05]{figures/AlphaStudy/GyroA13L05_Lines.png}
            \caption{}
        \end{subfigure}
        \begin{subfigure}{.19\textwidth}
            \centering
            \includegraphics[scale=.05]{figures/AlphaStudy/GyroA23L05_Lines.png}
            \caption{}
        \end{subfigure}
        \begin{subfigure}{.19\textwidth}
            \centering
            \includegraphics[scale=.05]{figures/AlphaStudy/GyroA33L05_Lines.png}
            \caption{}
        \end{subfigure}
    \end{subfigure}
    \caption{
        Comparison of different $\alpha$ values for a fixed $r_t=0.5$ between the different second frames in the double gyre field from \Cref{fig:energydevelopment}.
        The top row contains the footprints, the bottom row only the streamlines for visual clarity.
        (a) Field at $t=0$, the same first frame is used for (b) to (e).
        (b) $\alpha=0$, no coherence.
        (c) $\alpha=1/3$, some coherence on the image borders.
        (d) $\alpha=2/3$, good coherence in most regions, some strong changes remain.
        (e) $\alpha=1$, maximum coherence w/o regard for spatial placement.
    }
    \label[figure]{fig:pstudy1}
\end{figure}

\subsection{Parameter Study for $\alpha$ and $L_t$}
Since $L_t$ drives the temporal energy measure, and $\alpha$ determines its weight,
we show how different values for each parameter affect the images generated for two different datasets.
\begin{leftbar}
    We first introduce some variables for brevity and clarity: The radius of $L_t$ is written as $r_t$, and defined as a factor compared to the radius of $L_s$.
    To indicate that we use a radius for $L_t$ 1/3rd the size of $L_s$'s, we simply write $r_t=1/3$.
\end{leftbar}
The first set of images from \Cref{fig:pstudy1} uses $r_t = 0.5$ to not introduce artifacts from the footprints overlapping as discussed in \Cref{sec:tcmot}.
In (a), we see the baseline image of the first time step $t=0$, and (b) to (e) are all images of the second time step.
\paragraph*{\Cref{fig:pstudy1}(b)} With $\alpha=0$, the streamlines moved without regard for time coherence,
only achieving it by chance or seed choice.
\paragraph*{\Cref{fig:pstudy1}(c)} Using $\alpha=1/3$,
we immediately notice some improvements as the outer regions near the bottom and on the left are more stationary.
The footprints overall gain a more compact appearance, as green and yellow parts aren't as far apart as they were previously,
and we can see more regions where no streamlines were drawn before remain empty.
\paragraph*{\Cref{fig:pstudy1}(d)} $\alpha=2/3$ causes even the center to remain nearly the same, and the image seems to exhibit good time coherence all over.
An artifact at the top center was introduced due to the temporal preference, with a very short streamline appearing.
\paragraph*{\Cref{fig:pstudy1}(e)} The image degradation from the previous step is exacerbated;
with $\alpha=1$ we can see the spatial quality drop decisively when looking at the streamline representation compared to (c).
This is to be expected, as setting $\alpha=0$ causes the algorithm to disregard $E_t$ entirely, and setting $\alpha=1$ does the same to $E_s$.

The optimal range therefore lies between $[1/3 , 2/3]$, with the former possessing better spatial quality.
We move to the next set of pictures with $\alpha = 0.5$.

\newpage

\begin{figure}[ht]
    \centering
    \begin{subfigure}{\textwidth}
        \begin{subfigure}{.19\textwidth}
            \centering
            \includegraphics[scale=.05]{figures/AlphaStudy/GyroL1.png}
        \end{subfigure}
        \begin{subfigure}{.19\textwidth}
            \centering
            \includegraphics[scale=.05]{figures/AlphaStudy/GyroA12L005.png}
        \end{subfigure}
        \begin{subfigure}{.19\textwidth}
            \centering
            \includegraphics[scale=.05]{figures/AlphaStudy/GyroA12L05.png}
        \end{subfigure}
        \begin{subfigure}{.19\textwidth}
            \centering
            \includegraphics[scale=.05]{figures/AlphaStudy/GyroA12L1.png}
        \end{subfigure}
        \begin{subfigure}{.19\textwidth}
            \centering
            \includegraphics[scale=.05]{figures/AlphaStudy/GyroA12L3.png}
        \end{subfigure}
    \end{subfigure}
    \begin{subfigure}{\textwidth}
        \begin{subfigure}{.19\textwidth}
            \centering
            \includegraphics[scale=.05]{figures/AlphaStudy/GyroL1_Lines.png}
            \caption*{$r_t=1$}
        \end{subfigure}
        \begin{subfigure}{.19\textwidth}
            \centering
            \includegraphics[scale=.05]{figures/AlphaStudy/GyroA12L005_Lines.png}
            \caption*{$r_t=0.05$}
        \end{subfigure}
        \begin{subfigure}{.19\textwidth}
            \centering
            \includegraphics[scale=.05]{figures/AlphaStudy/GyroA12L05_Lines.png}
            \caption*{$r_t=0.5$}
        \end{subfigure}
        \begin{subfigure}{.19\textwidth}
            \centering
            \includegraphics[scale=.05]{figures/AlphaStudy/GyroA12L1_Lines.png}
            \caption*{$r_t=1$}
        \end{subfigure}
        \begin{subfigure}{.19\textwidth}
            \centering
            \includegraphics[scale=.05]{figures/AlphaStudy/GyroA12L3_Lines.png}
            \caption*{$r_t=3$}
        \end{subfigure}
    \end{subfigure}
    \caption{
        Comparison of different $\alpha$ values between the first five steps for an unsteady version of the double gyre field from \cref{fig:energydevelopment}.
        Each row represents five time steps and for each step, the left vortex moves down by $3\%$ of the image height.
        (a): With $\alpha=0$ most streamlines have low coherence and field movement is hard to notice between frames.
        (b): $\alpha=1/3$, TBD
        (c): $\alpha=2/3$, Good coherence in most regions, some strong changes remain
        (d): $\alpha=1$, Even better coherence with only slight changes, even in areas of field movement.
    }
    \label[figure]{fig:rchange}
\end{figure}

For the second part of this section, we look at how changes to $r_t$ affect the image as seen in \Cref{fig:rchange}.
We start with an extreme case for $r_t = 0.05$, giving us footprints
that are almost invisible in the rendered version of the low-pass image.
Accordingly, the footprints have almost no effect on the streamline placement,
and using low values for $r_t$ effectively removes time coherence altogether.
As the next example, we use an $r_t$ equal to the radius of $L_s$.
Here, we can see a lot of yellow regions, however, when using such a large radius,
they do not necessarily mean that time coherence was achieved.
We conclude search for a good $r_t$ with a final extreme case of $r_t=3$.
While almost the entire image is rendered as yellow due to the energy being capped at 2.0 when rendering
to fit the range [0, 255], the energy measure itself can use higher values.
In fact, the streamlines themselves exhibit time coherence regardless,
as their energies can still be compared and keep reacting to small
changes in position as long as they happen within the falloff distance due to the rasterization routine.
This reduces the accuracy however,
as contributions from one streamline can fade and be replaced by those from a different streamline more easily.

We thus conclude that there is a lower bound for $r_t$ after which time coherence is not reliably achieved.
An upper bound was not noticed, just a slow degradation of time coherence after the $r_t=1$ mark.
We also note the partial similarity between $\alpha$ and $r_t$, as placements generated using lower values for $r_t$
are virtually indistinguishable from those created with low $\alpha$s,
as both lead to a diminished final time coherence component in the total energy measure.

We therefore choose $\alpha = 0.5$ and $r_t=0.5$ as good starting values that introduce
a bias toward time coherence, while not regressing image quality too much and remaining
local enough to be effective and mostly guide the streamlines next to them.

\newpage


\begin{figure}[ht!]
    \centering
    \begin{subfigure}[b]{.3\textwidth}
        \centering
        \includegraphics[scale=.08]{figures/Shatter/3Lines.png}
        \caption{}
    \end{subfigure}
    \begin{subfigure}[b]{.3\textwidth}
        \centering
        \includegraphics[scale=.08]{figures/Shatter/15Lines.png}
        \caption{}
    \end{subfigure}
    \begin{subfigure}[b]{.3\textwidth}
        \centering
        \includegraphics[scale=.08]{figures/Shatter/3LinesRejoin.png}
        \caption{}
    \end{subfigure}
    \caption{
        (a) Three streamlines after being optimized.
        (b) To make the individual shards visible, we change the field's amplitude slightly.
        The shards' seeds are the center of the streamlines in (b), and all lie on one of the streamlines in (a).
        (c) Shards quickly rejoin when redrawn in the same field they originate from (a).
        We used a slightly reduced thickness to make it more visible that the shards
        do not simply overlap but actually rejoin into their single former streamline.
    }
    \label[figure]{fig:rejoin}
\end{figure}


\subsection{Shattering}
At the end of a time step's optimization phase, we break every streamline apart into smaller streamlets we refer to as \textit{shards}.
We start by dividing the parent streamline length-wise into sections which equal the length of the starting streamlines.
The shards are assigned a seed in the middle (lengthwise) of these intervals,
and their length equals the streamline start length.
If the parent streamline has some length remaining because it was not perfectly divisible by the start length, the last shard's length will be shorter.
This leaves each former streamline with the appearance of being dashed with each fragment having its own seed, and can be seen in \cref{fig:rejoin} (b).
The shards then act as the initial seeding strategy for the subsequent timeframe; the regular grid is only used for the first frame.
This way, we obtain many seeds that, if the field does not change too much, will merge back into the streamline they came from, as can be seen in \cref{fig:rejoin} (a) and (c),
saving iterations that would be needed for new seeding and lengthening in these regions.
If the field \textit{does} change, some segments will still reconnect and therefore keep their temporal coherence,
whereas areas of strong fluctuation will connect to different streamlines.
This results in changes being limited to parts where a streamline change is necessary,
providing extra streamlines in these areas while not affecting streamline trajectory too much on a global level.
\newpage

% \begin{figure}[ht]
%     \centering
%     \begin{subfigure}[b]{.32\textwidth}
%         \centering
%         \includegraphics[scale=.08]{figures/TBGyro.png}
%         \caption{}
%     \end{subfigure}
%     \begin{subfigure}[b]{.32\textwidth}
%         \centering
%         \includegraphics[scale=.08]{figures/TBGyro.png}
%         \caption{}
%     \end{subfigure}
%     \begin{subfigure}[b]{.32\textwidth}
%         \centering
%         \includegraphics[scale=.08]{figures/TBGyro.png}
%         \caption{}
%     \end{subfigure}
%     \caption{
%         (a): Two time steps without shattering.
%         (b): Two time steps with shattering.
%         (c): Total energy vs iteration step for (a) and (b).
%     }
%     \label[figure]{fig:combined}
% \end{figure}

\subsection{Combining Shattering and Coaxing}
Combining shattering and coaxing, we obtain a more reliable way of generating streamlines according to the footprint left behind by the last frame.
The seeds created during the shatter process all lie inside the footprint left behind by the previous streamline path.
Due to the coaxing function of the modified energy measure, it is unlikely that they will leave this valley solely due to the random movements of relaxation.
A change in the field is necessary in order to overcome the weight of the time coherence, making the streamline move or grow outside the previous footprint.
Due to the seeds being held in place in this way, it is very likely for them to rejoin to form the same lane they originated from.
If the field changes drastically in this region, the seeds can not fully connect to each other anymore, and will instead gravitate to a different footprint,
forming long patches of coherent streamlines whenever possible while still allowing relaxation to ensure good spatial distribution.\\
% In \cref{fig:combined} (c),
% we see that joining shards to form the streamlines is significantly faster than generating them anew.





% We have chosen to keep most of these actions as-is, the only difference introduced is
% a change to how the lengthening and shortening is done. 
% Instead of the two binary choices of lengthen/shorten and front/back, which only add/subtract a tiny bit at a time,
% we decided to choose a segment count at random between -5 and 5 for each end.
% This allows faster growth/shrinking (and hence faster convergence) while still preventing overlaps.

% \subsection{Initial Seeding}
% We prepare the image for the optimization routine by adding many streamlets with seeds on a regular grid to the image.
% This can also be done randomly yielding similar image quality,
% however strided access is more efficient with little to no benefit for the latter.

% \subsection{Oracle}
% The oracle from Turk and Bank's algorithm is used to suggest shorten/lengthen and move operations.
% Our oracle focuses on shorten/lengthen suggestions only.

% \subsection{Adding time coherence}
% We added two important modifications to the aforementioned algorithm to make it partially time-coherent.
% The first modification affects how seeds are chosen in the beginning of an optimization pass; the second affects
% how the energy measure is computed and lines are guided toward their final positions.
