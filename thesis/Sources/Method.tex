\chapter{Method}
At first, a heuristic criterion for temporal coherence between stream lines is defined.
Then ...

\section{Temporal Coherence}
Temporal Coherence refers to how a vector field behaves through different time steps.
Intuitively, we consider areas within the field to be of high temporal coherence if the lines drawn on them are relatively stationary.
Vice versa, we can say that an area of high fluctuation will be of low temporal coherence.
A more formal definition employed in our algorithm is as follows:
Given a field $F$ and a starting point $S_0$ (called the "seed"), we can integrate over the field.
This yields a set of points $S^0$ which define a streamline containing every reached point, written as $S^0 = \int(S_0, F)$.
We can therefore assign a streamline to every point in our field (and vice versa).
Given $S_0$ and an unsteady field $F(t)$, compute for each timestep $t_1...t_n$ the streamline $S^{0,t_i} = \int(S_0, F(t_i))$.
In order to convert these sets of lines to a scalar, we use the Hausdorff Distance $dist(S^i,S^j)$,
giving us the greatest minimal distance between any pair of two sets.
We can therefore create a map $coh(S_i, F(t)): max(dist(\int(S_i, F(t_k)), \int(S_i, F(t_l))))$,
sending each point in an unsteady vector field to a scalar, and thereby determining its temporal coherence.